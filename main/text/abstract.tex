\thispagestyle{empty}

  \begin{center}

    \textbf{Федеральное государственное автономное образовательное учреждение \\
высшего образования} \\
\textbf{«Российский университет дружбы народов имени Патриса Лумумбы»}

\hfill

\textbf{АННОТАЦИЯ}

\textbf{выпускной квалификационной работы }

\vspace{1cm}

С. Э. Косолапов

\vspace{1cm}
на тему: Имитационное моделирование самоорганизующихся сетей
\end{center}

\vspace{1cm}


Выпускная квалификационная работа посвящена имитационному
моделированию сетей VANET с использованием инструментов моделирования городской мобильности SUMO и сетевого симулятора NS-2. Целью работы является сравнение протоколов маршрутизации AODV, DSDV и DSR в подвижной автомобильной самоорганизующейся сети со стационарными узлами. А так же разработка необходимого программного комплекса для удобного проведения имитационного моделирования сетей VANET в различных условиях.

В результате работы было получено готовое окружение, подходящее для запуска имитационных экспериментов в сетях VANET. Моделирование мобильности было проведено с помощью симулятора SUMO и Open Street Map, а сетевые взаимодействия производились с помощью программы для NS-2. С помощью Makefile была удобным образом организована интеграция этих инструментов между собой на разных этапах моделирования. Визуализация результата производилась с помощью программы на языке Python и библиотеки matplotlib. Полученные графики были проанализированы и сделан вывод о лучшем протоколе маршрутизации для сетей VANET в описанных условиях.

\hfill

Автор ВКР \hspace{1cm} \underline{\hspace{3cm}} \hspace{1cm} \underline{С. Э. Косолапов }
