\subsection{Конфигурация сети}

Настройка сети в среде NS-2 включает в себя установку параметров, которые определяют характеристики симулируемой сети, включая типы устройств, параметры симуляции и конфигурации сетевых протоколов. Эти параметры являются ключевыми элементами для точного моделирования сетевых условий и оценки производительности сети. Информация по синтаксису была взята с сайта~\cite{ns2_docs}.

Далее рассмотрим элементы, которые нужно настроить.

\paragraph{Тип канала и модель радиораспространения.}
Используется беспроводной канал \linebreak(Channel/WirelessChannel) с моделью двулучевого земного распространения\linebreak(Propagation/TwoRayGround). Это указывает на физический уровень взаимодействия внутри сети, включая моделирование распространения сигнала и взаимодействия с окружающей средой.

\paragraph{Конфигурация сетевого интерфейса и MAC.}
Сетевой интерфейс определен как\linebreak Phy/WirelessPhy, а тип MAC-протокола соответствует стандарту IEEE 802.11\linebreak(Mac/802\_11). Это отражает распространенные стандарты беспроводной связи при моделировании поведения устройств в сети.

\paragraph{Очередь интерфейса и модель связи.}
Очередь интерфейса установлена как приоритетная очередь с механизмом DropTail (Queue/DropTail/PriQueue), а для связи между узлами используется тип связи на уровне канала (LL). 
Это определяет методы управления трафиком и перемещения данных в сети.

\paragraph{Антенна и параметры симуляции.}
Выбрана всенаправленная антенна\linebreak(Antenna/OmniAntenna) для устройств. Параметры среды симуляции включают размер области (956x600 м), максимальное число пакетов в очереди (50) и время окончания симуляции (200 секунд). Это определяет физические и временные рамки моделирования.

\paragraph{Конфигурация через аргументы командной строки.} В скрипте предусмотрена возможность изменения некоторых параметров (количество передвижных узлов, входные и выходные файлы, протокол маршрутизации) через аргументы командной строки. Это обеспечивает гибкость и настройку сценария симуляции под конкретные требования.

\paragraph{Динамическая адаптация настроек.} В зависимости от выбранного протокола маршрутизации (\verb|-rp|), тип очереди может быть изменен для оптимизации производительности сети в соответствии с выбранным алгоритмом.

Как это будет выглядеть в коде, показано в листинге~\ref{listing:ns2-net-config}.

\begin{lstlisting}[language=tcl, style=mystyle, caption=Конфигурация сети в NS-2, label=listing:ns2-net-config]
  set val(chan)   Channel/WirelessChannel    ;
  set val(prop)   Propagation/TwoRayGround   ;
  set val(netif)  Phy/WirelessPhy            ;
  set val(mac)    Mac/802_11                 ;
  set val(ifq)    Queue/DropTail/PriQueue    ;
  set val(ll)     LL                         ;
  set val(ant)    Antenna/OmniAntenna        ;
  set val(ifqlen) 50                         ;
  set val(sn)     6                          ;
  set val(x)      956                        ;
  set val(y)      600                        ;
  set val(stop)   200.0                      ;
  
  $val(netif) set Pt_ 0.1
  
  set argc [llength $argv]
  for {set i 0} {$i < $argc} {incr i} {
      set arg [lindex $argv $i]
      if {$arg == "-n"} {
          incr i
          set val(mn) [lindex $argv $i]
          continue
      }
      if {$arg == "-f"} {
          incr i
          set val(src) [lindex $argv $i]
          continue
      }
      if {$arg == "-o"} {
          incr i
          set val(out) [lindex $argv $i]
          continue
      }
      if {$arg == "-rp"} {
          incr i
          set val(rp) [lindex $argv $i]
          continue
      }
  }
  
  if { $val(rp) == "DSR" } {
      set val(ifq) CMUPriQueue
  } else {
      set val(ifq) Queue/DropTail/PriQueue
  }
  
  set val(nn) [expr $val(sn) + $val(mn) ]
\end{lstlisting}
  
  Далее идёт определение всех необходимых глобальных объектов (см. листинг~\ref{listing:ns2-globals}):
  
\begin{enumerate}
    \item Создание объекта симулятора NS-2.
    \item Инициализация GOD (General Operations Director).
    \item Открытие файла трассировки NS-2.
    \item Открытие файла трассировки NAM.
    \item Создание беспроводного канала.
\end{enumerate}

\newpage
  
\begin{lstlisting}[language=tcl, style=mystyle, caption=Определение глобальных объектов в NS-2, label=listing:ns2-globals]
  set ns_ [new Simulator]
  
  set topo       [new Topography]
  $topo load_flatgrid $val(x) $val(y)
  create-god $val(nn)
  
  set tracefile [open $val(out)/trace.tr w]
  $ns_ trace-all $tracefile
  
  set namfile [open $val(out)/trace.nam w]
  $ns_ namtrace-all $namfile
  $ns_ namtrace-all-wireless $namfile $val(x) $val(y)
  set chan [new $val(chan)];
\end{lstlisting}

Затем необходимо определить параметры мобильных узлов сети (см. листинг~\ref{listing:ns2-node-config}).

Проводится настройка параметров мобильного узла с использованием команды \verb|node-config|, предоставляемой экземпляром симулятора (обычно обозначаемым как \verb|$ns_|). В рамках данной настройки устанавливаются параметры, такие как алгоритм маршрутизации ad-hoc (\verb|-adhocRouting|), типы данных (например: \verb|-llType|, \verb|-macType|, \verb|-ifqType| и т.д.), а также параметры физического слоя (\verb|-phyType|), канала связи (\verb|-channel|), и топологии (\verb|-topoInstance|). Дополнительно активируются опции трассировки (\verb|-agentTrace|, \verb|-routerTrace|, \verb|-macTrace|, \verb|-movementTrace|), позволяющие следить за различными аспектами симуляции.

Далее в коде следует цикл \verb|for|, который используется для создания определенного количества узлов в симуляции. Количество создаваемых узлов указывается переменной \verb|$val(nn)|. Для каждого узла создается экземпляр с помощью вызова метода \verb|$ns_ node| и последующего сохранения полученного узла в массив \verb|node_|, индексируемый переменной цикла \verb|i|.

В заключение происходит импорт событий передвижения узлов из файла mobility.tcl.

\begin{lstlisting}[language=tcl, style=mystyle, caption=Настройка узлов сети, label=listing:ns2-node-config]
$ns_ node-config -adhocRouting  $val(rp) \
  -llType        $val(ll) \
  -macType       $val(mac) \
  -ifqType       $val(ifq) \
  -ifqLen        $val(ifqlen) \
  -antType       $val(ant) \
  -propType      $val(prop) \
  -phyType       $val(netif) \
  -channel       $chan \
  -topoInstance  $topo \
  -agentTrace    ON \
  -routerTrace   ON \
  -macTrace      ON \
  -movementTrace ON

for {set i 0} {$i < $val(nn) } {incr i} {
set node_($i) [$ns_ node]	
}

source $val(src)
\end{lstlisting}


\subsection{Создание стационарных узлов}

Так как сети VANET предполагают взаимодействие автомобилей со стационарными устройствами, то в рассматриваемом эксперименте надо добавить несколько узлов сети, которые не будут менять свое местоположение. Их необходимо задать вне файла mobility.tcl, в котором определяются все подвижные узлы сети (см. листинг~\ref{listing:ns2-nodes-define}).

\begin{lstlisting}[language=tcl, style=mystyle, caption=Настройка узлов сети, label=listing:ns2-nodes-define]

set src_node_i $val(mn)
set sink_node_i [expr $src_node_i + 5]

$node_($src_node_i) set X_ 420
$node_($src_node_i) set Y_ 190
$node_($src_node_i) set Z_ 0

$node_([expr $src_node_i + 1]) set X_ 300
$node_([expr $src_node_i + 1]) set Y_ 200
$node_([expr $src_node_i + 1]) set Z_ 0

$node_([expr $src_node_i + 2]) set X_ 155
$node_([expr $src_node_i + 2]) set Y_ 215
$node_([expr $src_node_i + 2]) set Z_ 0

$node_([expr $src_node_i + 3]) set X_ 130
$node_([expr $src_node_i + 3]) set Y_ 345
$node_([expr $src_node_i + 3]) set Z_ 0

$node_([expr $src_node_i + 4]) set X_ 290
$node_([expr $src_node_i + 4]) set Y_ 360
$node_([expr $src_node_i + 4]) set Z_ 0

$node_($sink_node_i) set X_ 115
$node_($sink_node_i) set Y_ 490
$node_($sink_node_i) set Z_ 0

for {set i 0} {$i < $val(nn) } {incr i} {
    $ns_ initial_node_pos $node_($i) 10
}
\end{lstlisting}


\subsection{Генерация трафика}

Финальным этапом настройки сценария симуляции является настройка генерации трафика. В проводимом эксперименте предлагается генерировать трафик из одного стационарного узла в другой. Подвижные же узлы сети будут влиять на маршруты, выбираемые алгоритмами маршрутизации во время проведения симуляции.

Для генерации трафика в NS-2 необходимо назначить узлам TCP агентов (Agent/TCP/Newreno для источника и Agent/TCPSink для узла назначения) и добавить приложение, которое и будет гегнерировать трафик (Application/FTP). Код для генерации трафика отображён в листинге~\ref{listing:ns2-traffic}.

\begin{lstlisting}[language=tcl, style=mystyle, caption=Генерация трафика, label=listing:ns2-traffic]
set tcp_0 [new Agent/TCP/Newreno]
$ns_ attach-agent $node_($src_node_i) $tcp_0
set sink_0 [new Agent/TCPSink]
$ns_ attach-agent $node_($sink_node_i) $sink_0
$ns_ connect $tcp_0 $sink_0
$tcp_0 set packetSize_ 1500

set ftp_0 [new Application/FTP]
$ftp_0 attach-agent $tcp_0
$ns_ at 0.1 "$ftp_0 start"
$ns_ at $val(stop) "$ftp_0 stop"

\end{lstlisting}

\subsection{Запуск симуляции}

После завершения настройки необходимо написать код для запуска симуляции и предусмотреть её корректное завершение (см. листинг~\ref{listing:ns2-finish}).

\begin{lstlisting}[language=tcl, style=mystyle, caption=Запуск сценария и завершение, label=listing:ns2-finish]
proc finish {} {
  global ns_ tracefile namfile
  $ns_ flush-trace
  close $tracefile
  close $namfile
  exit 0
}
for {set i 0} {$i < $val(nn) } { incr i } {
    $ns_ at $val(stop) "$node_($i) reset"
}
$ns_ at $val(stop) "$ns_ nam-end-wireless $val(stop)"
$ns_ at $val(stop) "finish"
$ns_ at $val(stop) "puts \"done\" ; $ns_ halt"
$ns_ run
\end{lstlisting}

Когда код программы готов --- можно запускать симуляцию через NS-2.

На выходе из симуляции получается трейсфайл с описанием всех событий, происходивших во время симуляции. В нём содержится вся информация о передаваемых пакетах и о том, в какой момент времени в какой узел пакет поступил.

Знание модели позволяет подходящим образом настроить окружение для проведения экспериментов, а так же автоматизировать процесс, где это возможно.
 