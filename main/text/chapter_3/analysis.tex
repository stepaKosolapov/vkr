\subsection*{Возникшие проблемы}

В процессе неоднократных попыток запуска эксперимента протоколы DSR и AODV стабильно показывали похожий на реальные значения результат, а именно - отличные от нуля параметры сети, такие как количество отправленных пакетов, процент доставки пакетов, задержка и тд. Но была одна проблема - DSDV не работал в поставленных условиях симуляции. 

Это проявлялось в том, что при запуске симуляции с использованием протокола DSDV, стационарные узлы не успевали начать отправлять пакеты. Возникло предположение, что это происходит из-за постоянно и быстро меняющейся топологии сети. Конкретно - узлы двигались с большой скоростью из-за чего не успевала построиться таблица маршрутизации, необходимая для работы DSDV.

Решением данной проблемы стало добавление вот новых строк в код программы, производящих настройку агента DSDV. 

По сути эти настройки увеличивают производительность и частоту построения таблицы DSDV.

Необходимо пояснить эти настройки.

Строка `Agent/DSDV set perup\_ 6 ;`

Эта строка устанавливает значение параметра perup\_ для агента DSDV равным 6. Параметр perup\_ относится к периодичности (времени в секундах), с которой DSDV генерирует полные объявления маршрутизации. В данном случае, полные объявления будут генерироваться каждые 6 секунд. Это позволяет узлам обновлять информацию о маршрутизации в своих таблицах маршрутизации с указанной периодичностью.

Строка `Agent/DSDV set use\_mac\_ 0 ;`

Здесь устанавливается параметр use\_mac\_ равным 0 для агента DSDV. Параметр use\_mac\_ определяет, будет ли для передачи сообщений объявлений DSDV использоваться MAC (Media Access Control) адресация. Значение 0 указывает, что MAC адресация не будет использоваться. В контексте симуляции это может быть полезно для упрощения модели или когда не требуется моделирование на уровне доступа к среде передачи данных. По сути это сильно улучшает производительность и ускоряет симуляцию.

Строка `Agent/DSDV set min\_update\_periods\_ 2 ;`

Эта строка устанавливает минимальное количество периодов между отправками полных объявлений, параметр min\_update\_periods\_, равным 2 для агента DSDV. Это значит, что между consecutive (последовательными) полными объявлениями должно пройти не менее двух периодов выпуска этих объявлений. Это может быть использовано для ограничения количества сетевого трафика, вызванного передачей объявлений, путем установления минимального интервала времени между объявлениями.

\begin{lstlisting}[language=tcl, style=mystyle, caption=Дополнительные настройки для Agent/DSDV]
if { $val(rp) == "DSDV" } {
    Agent/DSDV set perup_          6        ;
    Agent/DSDV set use_mac_        0        ;
    Agent/DSDV set min_update_periods_ 2    ;
}
\end{lstlisting}

Кроме того, для уменьшения скорости передвижения узлов была использована программа NetEdit. Она позволяет задать ограничение для симуляции мобильности и отредактировать "карту местности"\, которая будет отправлена в SUMO. С помощью этого инструмента была уменьшена максимальная скорость на рассматриваемом участке местности, чтобы увеличить время нахождения узлов в сети и расширить окно, в котором существует способ передавать данные между двумя узлами.

\subsection*{Сравнение протоколов}

После всех правок все три протокола стали давать ожидаемые метрики и визуально(через NetAnim) стало возможно наблюдать передачу данных в момент, когда подвижные узлы создают сеть, по которой становится возможно передать данные из одного стационарного узла в другой.

Настроенное окружение позволяет сразу создать и проанализировать графики зависимости параметров сети от количества узлов в сети для всех рассматриваемых протоколов маршрутизации.

Параметры сети были выбраны следующие: 

\begin{enumerate}
    \item PDR(Packet Delivery Ratio) - процент доставки пакетов.
    \item Delay - время, требующеся для доставки пакета внутри сети.
    \item Packets sent - количество отправленных пакетов внутри сети.
    \item NRL(Normalized Routing Load) - метрика, показывающая сколько пакетов из всех отправленных служит целям роутинга, а не передаче данных.
    \item Throughput - пропускная способность сети.
\end{enumerate}

\subsection*{Метрика End-to-end delay}

\begin{figure}[!h]
    \centering
    \includegraphics[width=1\linewidth]{"delay.png"}
    \caption{Параметр сети Delay}
    \label{fig:delay_plot}
\end{figure}

По рис.~\ref{fig:delay_plot} видно, что лучше всех показатель задержки у протокола AODV и в зависимости от количества узлов в сети положение вещей не меняется. Второе место занимает DSDV, он не так сильно отстаёт от AODV, но в поставленном эксперименте немного проигрывает. Протокол DSR в параметре задержки сильно уступает двум другим протоколам. В зависимости от количества узлов в сети задержка для протокола DSR уменьшается довольно сильно, возможно играет роль плотность узлов в сети.
  

\subsection*{Метрика NRL}

Исходя из рис.~\ref{fig:nrl_plot}, параметр NRL закономерно выше у AODV и DSDV, т.к достаточно большое количество пакетов отправленных в сеть служат только для нахождения необходимого пути и построения таблиц маршрутизации. Для DSR такой проблемы нет, что обсуловлено устройством DSR. Видно, что AODV и DSDV сильно не отличаются по этому параметру, а DSR заметно и кратно выигрывает. Так же понятно и подтверждается, что при увеличении количества узлов - процент пакетов, отправленных для целей маршрутизации в протоколах DSDV и AODV сильно увеличивается, а у DSR остаётся примерно на прежнем уровне, лишь показывая очень слабый, но закономерный рост.

\begin{figure}[!h]
    \centering
    \includegraphics[width=1\linewidth]{"nrl.png"}
    \caption{Параметр сети NRL}
    \label{fig:nrl_plot}
\end{figure}

\subsection*{Метрика Packets sent}

Исходя из рис.~\ref{fig:packets_sent_plot}, количество отправленных пакетов в сеть у протоколов DSR и DSDV заметно выше чем у протокола AODV. Видно, что у протокола DSDV при увеличении количества узлов количество отправленных в сеть пакетов сильно возрастает, когда у протокола DSR после определённого количества узлов остаётся примерно неизменной. У протокола AODV в целом от количества узлов сильно не меняется количество отправленных пакетов и остаётся на стабильно низком уровне.

\begin{figure}[!h]
    \centering
    \includegraphics[width=1\linewidth]{"packets_sent.png"}
    \caption{Параметр сети Packets sent}
    \label{fig:packets_sent_plot}
\end{figure}

\subsection*{Метрика PDR}

Исходя из рис.~\ref{fig:pdr_plot}, процент доставки пакетов заметно лучше у протокола AODV. Он начинается от 80\% и стабильно растёт при увеличении количества узлов в сети. У протокола DSR ситуация по тенденции роста похожая, но рост не так очевиден и в целом средний уровень процента доставки пакетов ниже примерно в два раза чем у AODV. Интересная ситуация происходит с протоколом DSDV. При увеличении количества узлов этот протокол показывает очень плохие результаты по доставке пакетов, вплоть до  10\%, что конечно по сравнению с AODV и DSR очень плохо. 
  
\begin{figure}[!h]
    \centering
    \includegraphics[width=1\linewidth]{"pdr.png"}
    \caption{Параметр сети PDR}
    \label{fig:pdr_plot}
\end{figure}

\subsection*{Метрика Throughput}

Исходя из рис.~\ref{fig:throughput_plot} пропускная способность самая большая оказалась у DSR и составляет в среднем больше 180 kbps. На втором месте протокол AODV со значением в районе 150 kbps. Протокол DSDV в параметре пропускной способности опять оказывается самым худшим из трёх. Пропускная способность при использовании DSDV опускается ниже 80 kbps при увеличении количества узлов. Кроме того, можно заметить, что у протоколов DSR и AODV практически не изменяется пропускная способность от изменения количества узлов и примерно остаётся на одном уровне, в то время как при использовании DSDV количество узлов влияет на пропускную способность. 

\begin{figure}[!h]
    \centering
    \includegraphics[width=1\linewidth]{"throughput.png"}
    \caption{Параметр сети Throughput}
    \label{fig:throughput_plot}
\end{figure}

На основе проведенного анализа результатов эксперимента, относящегося к параметрам сети таким как PDR (Packet Delivery Ratio), Delay (время доставки пакета), Packets Sent (количество отправленных пакетов), NRL (Normalized Routing Load), и Throughput (пропускная способность сети), можно сделать следующие выводы:

\begin{enumerate}
    \item Анализ метрики Delay показывает, что протокол AODV демонстрирует наилучшие показатели задержки, что сохраняется независимо от количества узлов в сети. Протоколы DSDV и DSR показывают более высокие значения задержки, с относительным улучшением показателей DSR при увеличении плотности сети.

    \item Исследование метрики NRL выявляет, что DSR значительно превосходит AODV и DSDV по эффективности использования сетевых ресурсов, продемонстрировав более низкий уровень роутинговой нагрузки. Это указывает на меньшее количество контрольных пакетов, необходимых для маршрутизации в протоколе DSR.

    \item Показатель общего количества отправленных пакетов (Packets sent) подчеркивает эффективность AODV в поддержании низкого уровня сетевого трафика, в то время как DSDV и DSR испытывают значительное увеличение числа отправляемых пакетов при росте сети, что особенно выражено у DSDV.

    \item Анализ метрики PDR демонстрирует, что AODV обеспечивает наивысший процент успешной доставки пакетов, что делает его предпочтительным выбором для обеспечения надежности сетевой передачи данных. DSR и DSDV показывают более низкие результаты, причем DSDV выявил значительное падение производительности при увеличении числа узлов.

    \item Оценка пропускной способности сети (Throughput) показывает, что, несмотря на наивысшие показатели у DSR, AODV обеспечивает сбалансированное сочетание производительности и надежности сети, что подчеркивается его хорошими результатами и в метрике Throughput, и в других рассмотренных параметрах.
\end{enumerate}

Итак, учитывая общую производительность в контексте рассмотренных метрик, протокол AODV демонстрирует себя как наиболее подходящий для использования в условиях проведенного эксперимента. Это обусловлено его способностью обеспечивать высокую пропускную способность, низкую задержку, а также высокий PDR, что в совокупности подтверждает его эффективность для обеспечения качественной и надежной сетевой коммуникации. Данный вывод коррелирует с результатами, полученными в исследовании, упомянутом в ~\cite{rizwan2018vanet}, подчеркивая превосходство AODV над другими рассмотренными протоколами в разнообразных сценариях использования.

В ~\cite{rizwan2018vanet} авторы обозревают и анализируют различные протоколы маршрутизации, чтобы выяснить, какой из них наилучшим образом подходит для применения в VANET, особенно для задач связанных с видео потоковой передачей. Основной вывод работы заключается в том, что протокол AODV показывает наилучшие результаты как в стандартных, так и в более сложных сценариях использования VANET, включая передачу видео, по сравнению с другими адаптивными сетевыми протоколами, такими как DSR. Как можно наблюдать по результатам нашего эксперимента - протокол AODV так же даёт лучший результат почти что по всем параметрам относительно других протоколов.