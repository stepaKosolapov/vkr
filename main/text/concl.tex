\chapter*{Заключение}

В ходе выполнения выпускной квалификационной работы бакалавра достигнуты поставленные цели и решены ключевые задачи, направленные на исследование и анализ эффективности протоколов маршрутизации в сетях VANET с использованием инструментов моделирования SUMO и NS-2. Было проведено подробное сравнение трех протоколов маршрутизации: DSDV, AODV и DSR, с целью выявления их эффективности в условиях, когда стационарные узлы взаимодействуют с использованием подвижных узлов, формирующих динамичную автомобильную сеть.

Основные задачи, успешно реализованные в рамках данной работы, включали в себя:

\begin{enumerate}
    \item Составление детального описания имитационной модели, предназначенной для анализа протоколов маршрутизации в сетях VANET. Это обеспечило необходимую основу для дальнейшего исследования и анализа.
    \item Разработка программного комплекса, способствующего проведению экспериментов с сетями VANET в разнообразных условиях. Разработанный комплекс оказал значительную поддержку в анализе и сравнении протоколов маршрутизации, что является ключевой целью работы.
    \item Выполнение аналитического анализа результатов экспериментов. В результате анализа были получены выводы относительно эффективности исследуемых протоколов маршрутизации, что способствовало глубокому пониманию работы и особенностей сетей VANET в представленных условиях.
\end{enumerate}

Выполненная работа продемонстрировала анализ выбранных протоколов маршрутизации, их практическое применение в условиях сетей VANET и способность адаптации к специфическим условиям этих сетей. Результаты исследования могут быть использованы в качестве основы для дальнейшего развития исследований в области интеллектуальных транспортных систем и сетей VANET, способствуя развитию эффективных методов маршрутизации и повышению качества коммуникационных процессов в автомобильных сетях.

Таким образом, достигнутые результаты подтверждают актуальность и значимость исследованной проблематики, важность расширения знаний и разработки новых подходов в области моделирования и анализа сетей VANET.