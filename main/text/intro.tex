\chapter*{Введение}

Работа посвящена имитационному моделированию сетей VANET с использованием инструментов моделирования SUMO и NS-2. 
Объектом исследования были выбраны протоколы маршрутизации, использующиеся в сетях VANET: DSDV, AODV и DSR. 
Предметом исследования является сравнение эффективноси данных трёх протоколов маршрутизации в определённых условиях, когда стационарные узлы, находящиеся вне зоны доступа друг от друга, пытаются отправлять данные, используя лишь подвижные узлы, образующие самоорганизующую автомобильную сеть. 
В рамках работы было проведено подробное описание имитационной модели, а так же был разработан программный комплекс для удобного проведения экспериментов с сетями VANET в различных условиях. Был проведён анализ полученных графиков и сделан вывод об эффективности протоколов маршрутизации сетей VANET в описанных условиях.

\subsection*{Актуальность}

Изучаемая тема актуальна по причине развития интеллектуальных транспортных систем в мире и России, а так же из-за небольшого количества актуальных русскоязычных материалов по имитационному моделированию сетей VANET.

\subsection*{Цель работы}

Целью выпускной квалификационной работы является определение эффективности протоколов маршрутизации применяемых в самоорганизющихся сетях в условиях проводимого эксперимента.

\subsection*{Задачи}

Основными задачами работы являются:

\begin{enumerate}
    \item Разработка программного комплекса, необходимого для проведения эксперимента с помощью средств моделирования NS-2 и SUMO.
    \item Проведение эксперимента по работе протоклов маршрутизации в условиях сети VANET со стационарными и подвижными узлами.
    \item Сравнительный анализ характеристик протоколов маршрутизации самоорганизющихся сетей DSDV, AODV и DSR.
\end{enumerate}

\subsection*{Структура работы}

Работа состоит из введения, трех разделов, заключения и списка используемой литературы. Во введении приведено краткое описание работы, обусловлена актуальность
работы, а также поставлена цель и сформулированы задачи выпускной квалификационной работы.

В первом разделе работы дано введение в интеллектуальные транспортные системы и описание особенностей VANET, как части интеллектуальных транспортных систем. Так же рассмотрено устройство протоколов маршрутизации, используемых в сетях VANET, их отличительные особенности.

Во втором разделе работы представлено описание имитационной модели, а так же описаны методологии, используемые для анализа результатов моделирования. Рассмотрены подробнее инструменты моделирования SUMO и NS-2, их особенности и отличия от аналогов.

В третьем разделе выпускной квалификационной работы рассмотрена структура итогового программного комплекса, подробно описаны элементы окружения эксперимента и части программного кода, необходимого для запуска эксперимента. Анализ результатов со сравнением протоколов маршрутизации так же представлен в третьем разделе работы.

В заключении подведены общие итоги работы, изложены основные выводы.

