\subsection{Суть эксперимента}

В данной работе будет проведён эксперимент, в рамках которого будет производится моделирование сети VANET. 

Схема эксперимента показана на рис.~\ref{fig:experiment_scheme}.

\begin{figure}[!h]
    \centering
    \includegraphics[width=1\linewidth]{"experiment.png"}
    \caption{Схема эксперимента}
    \label{fig:experiment_scheme}
\end{figure}

На какой-то местности расположены несколько стационарных узлов с беспроводным передатчиком связи (RSU). Мощность датчиков и расстояние между узлами таковы, что стационарные узлы не могут достать друг до друга и передать какие-либо данные непосредственно. 

Моделируемая местность имеет автомобильную дорогу, по которой передвигаются подвижные узлы с бортовым беспроводным передатчиком связи (OBU). Подвижные узлы передвигаются по правилам автомобильного дорожного движения на моделируемом участке дороги.

В какой-то момент симуляции, когда подвижные узлы изменят топологию сети таким образом, что между стационарными узлами образуется путь (включающий подвижные узлы), начнётся передача данных из одного стационарного узла в другой. При этом путь, по которому следуют данные, будет неизбежно меняться с течением времени симуляции, так как подвижные узлы почти непрерывно передвигаются и меняют топологию сети. Подвижные узлы могут так же пропадать из зоны доступности стационарных узлов.

В такой нестабильной сети, когда топология сети постоянно \textquote{рвётся}, предлагается проверить эффективность трёх протоколов маршрутизации: AODV, DSDV, DSR. Принципы моделирования сети и анализа эффективности протоколов будут рассмотрены в этой главе.