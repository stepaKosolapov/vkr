Как видно на рис.~\ref{fig:mobility_scheme}, после запуска симуляции NS-2 имеются два артефакта - трейс-файл движения пакетов и файл анимации.

Каждый из них можно проанализировать соответствующим способом. 

\subsection*{Визуальный анализ}

\begin{figure}[!h]
  \centering
  \includegraphics[width=1\linewidth]{"NetAnim_interface.jpeg"}
  \caption{Интерфейс NetAnim}
  \label{fig:netanim_interface}
\end{figure}

Первый и самый очевидный способ анализа - визуальный. Вместе с NS-2 в архиве "all-in-one" поставляется программа NetAnim. Это програма для визуального отображения процесса симуляции.

В скрипте NS-2, как уже было рассмотрено ранее, присутствует специальная команда namtrace-all, которая записывает все события в файл анимации. 

После запуска симуляции, с помощью NetAnim можно просмотреть анимацию из полученного файла в формате .nam.

Интерфейс NetAnim изображён на рис.~\ref{fig:netanim_interface}.

При запуске анимации - подвижные узлы сети начинают передвигаться и посылать пакеты. Узлы в интерфейсе изображены в виде круга, содержащего внутри себя номер узла, который был задан в скрипте NS-2. 

Так как используется беспроводной канал, широковещательные сообщения изображены расплывающимися окружностями от узла. 

Направленные пакеты к определённому узлу изображены отрезком, перемещающимся в пространстве от источника к приёмнику.

С помощью визуального наблюдения можно быстро определить те проблемы, которые труднее обнаружить с помощью анализа трейс-файла. 

Например, наблюдая за анимацией, можно обнаружить, что прямой связи между стационарными узлами не существует и пакеты начинают передаваться только когда между стационарными узлами появляется подвижный узел.

Пример кадра из анимации можно увидеть на рис.~\ref{fig:netanim_frame}.

\begin{figure}[!h]
  \centering
  \includegraphics[width=1\linewidth]{"NetAnim_frame.jpeg"}
  \caption{Кадр из анимации NetAnim}
  \label{fig:netanim_frame}
\end{figure}

\subsection*{Структура трейс-файла}

Чтобы проводить не визуальный анализ сети, необходимо разобраться в том, как устроен второй полученный артефакт - трейс-файл.

Для примера будет рассмотрена 16 строка из трейс-файла, часть содержимого которого можно видеть на рис.~\ref{fig:ns2_tracefile}.

Это позволит разобраться в том, как устроен трейс-файл и придумать принцип его анализа.

\begin{figure}[!h]
  \centering
  \includegraphics[width=1\linewidth]{"ns2_tracefile.png"}
  \caption{Содержимое трейса ns2}
  \label{fig:ns2_tracefile}
\end{figure}

r 10.000948296 \_38\_ RTR  --- 0 AODV 48 [0 ffffffff 0 800] ------- [0:255 -1:255 30 0] [0x2 1 1 [1 0] [0 4]] (REQUEST). Эта строка из трейс-файла NS-2 содержит информацию об одном пакете, который был отправлен в сеть. 
Расшифровка элементов строки:

\begin{enumerate}
  \item $r$ - тип события, в данном случае это отправка пакета.
  \item $10.000948296$ - время отправки пакета в секундах.
  \item $\_38\_$ - идентификатор узла, который отправил пакет.
  \item $RTR$ - тип узла, который отправил пакет (в данном случае это маршрутизатор).
  \item $---$ - идентификатор узла, который должен получить пакет (в данном случае это широковещательный адрес).
  \item $0$ - идентификатор канала, по которому был отправлен пакет.
  \item $AODV$ - протокол маршрутизации, который использовался для отправки пакета.
  \item $48$ - размер пакета в байтах.
  \item $[0 ffffffff 0 800]$ - информация о маршруте, который будет использоваться для доставки пакета. В данном случае это начальный пакет, поэтому маршрут не известен и все значения установлены в ноль. 
  \item $-------$ - информация о том, какими типами нод был пройден пакет. В данном случае это еще неизвестно, поэтому все значения установлены в ноль.
  \item $[0:255 -1:255 30 0]$ - информация о маршруте, который будет использоваться для доставки пакета. В данном случае это начальный пакет, поэтому маршрут не известен и все значения установлены в ноль.
  \item $[0x2 1 1 [1 0] [0 4]]$ - информация о пакете. $[0x2]$ - это тип пакета, $[1]$ - это идентификатор пакета, $[1]$ - это количество попыток отправки, $[1 0]$ - это номер сегмента, который был отправлен, $[0 4]$ - это номер сегмента, который будет отправлен.
  \item $(REQUEST)$ - тип пакета, в данном случае это запрос.
\end{enumerate}

Из трейс-файла можно понять состояние каждого пакета в любой момент времени симуляции: размер пакета, пункт назначения, маршрут и т.п.

Это знание позволяет провести анализ производительности сети, обработав и достав нужную информацию о пакетах из трейс-файла.

\subsection*{Подсчёт параметров сети}

Чтобы посчитать параметры сети зачастую используются скрипты в формате awk. 

Принцип написания этих файлов прост, каждая строка трейс-файла разделяется на части. Как было указано выше, данные в строке трейс-файла разделены пробелами, поэтому после разделения строки пробелами получается массив данных. Каждый элемент в зависимости от типа сообщения содержит разные данные. 

Предлагается рассчитать несколько интересных для исследования параметров сети:

\begin{enumerate}
  \item PDR(Packet Delivery Ratio) - процент доставки пакетов.
  \item Delay - время, требующеся для доставки пакета внутри сети.
  \item Throughput - пропускная способность сети.
  \item NRL(Normalized Routing Load) - метрика, показывающая сколько пакетов из всех отправленных служит целям роутинга, а не передаче данных.
  \item Packets sent - количество отправленных пакетов внутри сети.
\end{enumerate}

\subsection*{Packet Delivery Ratio}

Коэффициент успешной доставки пакетов (Packet Delivery Ratio, PDR) определяется как отношение числа пакетов, которые были успешно доставлены к месту назначения, к числу пакетов данных, отправленных из источника. Этот показатель вычисляется путем деления количества пакетов, полученных пунктом назначения, на количество пакетов, исходящих от источника. Высокий коэффициент PDR свидетельствует о лучшей производительности протокола.

\begin{equation}
  \label{eq:pdr_calculation}
  \text{PDR} = \frac{P_r}{P_s} \times 100(\%)
\end{equation}

Где $P_r$ это суммарное количество отправленных пакетов, а $P_s$ это суммарное количество принятых пакетов.

\subsection*{Delay}

Среднее время задержки определяется как общее временя от отправления пакета источником до его прибытия к пункту назначения.

\begin{equation}
  \label{eq:delay_calculation}
  \text{EED} = \frac{1}{n} \sum_{i=1}^{n}(T_{r_i} - T_{s_i}) \times 1000
\end{equation}

Где $i$ это идентификатор пакета, $n$ - число успешно доставленных пакетов, $T_{r_i}$ - время приёма пакета, а $T_{s_i}$ это время отправки пакета. 

\subsection*{Throughput}

Пропускная способность определяется как число пакетов с данными, которые были успешно доставлены в сети за единицу времени. Высокая пропускная способность свидетельствует о лучшей производительности протокола.

\begin{equation}
  \label{eq:throughput_calculation}
  \text{TH} = \frac{B_r \times 8}{T_{\text{stop}} - T_{\text{start}} \times 1000}
\end{equation}

Где $B_r$ это количество полученных битов, $T_{\text{start}}$ - время старта отправки, $T_{\text{stop}}$ - время конца отправки.

\subsection*{NRL} 

Основная цель NRL — измерить количество управляющих данных, необходимых для доставки пакетов данных от источника к получателю, нормализуя это значение на количество успешно доставленных пакетов данных. Это позволяет провести сравнение эффективности маршрутизации между различными протоколами или конфигурациями сети относительно использования сетевых ресурсов.

\begin{equation}
  \label{eq:nrl_calculation}
  \text{NRL} = \frac{P_{s_{\text{routing}}}}{P_{r_{\text{data}}}}
\end{equation}

Где $P_{s_{\text{routing}}}$ это количество отправленных управляющих пакетов(необходимых для обеспечения маршрутизации), а $P_{r_{\text{data}}}$ - количество успешно принятых пакетов с данными.

\subsection*{Packets sent}

Эта метрика отвечает за количество отправленных пакетов в сети и никакой особой формулы для вычисления не имеет, достаточно подсчитывать все отправленные пакеты.

