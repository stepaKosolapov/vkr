\subsection*{Моделирование сети}

Симулятор сети применяется для моделирования обмена сообщениями между соединенными узлами~\cite{aljabry2021survey}. В контексте VANET это обычно включает в себя бортовые устройства транспорта (OBU) и стационарные придорожные устройства (RSU) и в большинстве случаев связано с беспроводной коммуникацией. В идеале моделируются все компоненты коммуникационной системы (например, весь стек протоколов), и в конечном итоге симуляция также включает другие соответствующие метрики (например, отношение сигнал/шум, показатели ошибок пакетов). Модель сети описывает как компоненты сети, так и события. Узлы, маршрутизаторы, коммутаторы и связи являются примерами компонентов. События, в свою очередь, могут включать передачи данных и ошибки пакетов.

Для данного сценария симуляции выходные данные от симулятора сети обычно включают метрики, связанные с сетью, соединениями и устройствами. Также обычно доступны файлы трассировки. Такие файлы записывают каждое произошедшее событие в симуляции и могут быть обработаны для дальнейшего анализа. Большинство доступных симуляторов сетей базируются на дискретно-событийном моделировании. В этом подходе хранится список \textquote{ожидающих событий}, которые затем обрабатываются по порядку на каждом шаге симуляции. Некоторые события могут инициировать новые. Например, прибытие пакета на узел может спровоцировать отправку нового пакета. Примеры доступных симуляторов сетей (некоторые из них широко используются в сетях VANET) включают OMNeT++, OPNET, JiST/SWANS, NS-3 и NS-2.

В данной работе подробнее будет рассмотрен сетевой симулятор NS-2.

\subsection*{Сетевой симулятор NS-2}

Network Simulator 2 (NS-2) ~\cite{issariyakul2009introduction}, ~\cite{mahrenholz2004real} представляет собой программный комплекс на основе объектно-ориентированной парадигмы, разработанный для моделирования различных аспектов сетевых систем и процессов. Сердцевиной системы является ядро, написанное на языке программирования C++, а для сценариев и интерфейса пользователя применяется язык Object oriented Tool Command Language (OTcl). Наличие классовой структуры и иерархии, которая поддерживается обоими языками, позволяет удобно организовать моделирование, сохраняя при этом однозначное соответствие между элементами разработки на разных уровнях.

Платформа NS-2 обеспечивает поддержку широкого круга сетевых протоколов, что делает её весьма полезным инструментом в изучении и анализе современной сетевой инфраструктуры. Включает в себя реализации протоколов на всех уровнях сетевого обмена данных, таких как MPLS, IPv6, OSPF, и RSVP, а также несколько механизмов управления очередями (например, RED, WFQ, CBQ, SFQ), что позволяет имитировать работу реальных сетевых условий. Особое внимание уделено поддержке протоколов для беспроводных сетей, в том числе AODV, DSDV, DSR, расширяя область применения инструмента.

Среди ключевых особенностей NS-2 выделяется его гибкость в имитации различных типов трафика, включая трафик с Пуассоновским распределением и самоподобный трафик. Эта возможность дополнена способностью пользователя создавать собственные математические модели и функции, используя C++. К тому же, NS-2 позволяет моделировать ошибки на канальном уровне в процессе передачи данных, такие как искажение и потери пакетов, предлагая разнообразные методы для задания характеристик ошибок.

Для анализа и интерпретации результатов моделирования в NS-2 интегрированы инструменты визуализации, включая Network Animator (NAM) и Xgraph. NAM позволяет наглядно представить динамику сетевого взаимодействия, включая топологию, потоки данных и работы сетевого оборудования, на основе данных трассировки. Xgraph же облегчает анализ статистики сетевой активности, позволяя строить графики непосредственно изскриптов моделирования.

Программный комплекс NS-2, разрабатываемый как программное обеспечение с открытым исходным кодом, доступен для бесплатного использования, модификации и распространения. Это обеспечивает его высокую приспособляемость к специфическим исследовательским задачам и различным сетевым сценариям. Поддержка множества операционных систем, включая Linux, OS X, Solaris, FreeBSD и Windows, увеличивает удобство использования NS-2 в различных средах разработки и обеспечивает широкую доступность для пользователей, независимо от их предпочтений в операционных системах.

Важным аспектом, способствующим популярности и функциональному развитию NS-2, является активное сообщество разработчиков и пользователей. Это сообщество обеспечивает постоянное обновление и расширение возможностей программного комплекса, разработку новых модулей и протоколов, а также предоставление документации, руководств и разнообразных учебных материалов. Такой открытый подход способствует повышению качества и эффективности как самого программного обеспечения, так и проводимых с его помощью исследований.

\subsection*{Принцип моделирования сети в симуляторе NS-2}

В создании сценария симуляции в NS-2 можно выделить несколько этапов и частей.

\begin{enumerate}
  \item Конфигурация параметров сети и узлов
  \item Перемещение узлов сети
  \item Генерация трафика в сети
  \item Отслеживание событий и запись их в трейс-файлы
  \item Завершение симуляции
\end{enumerate}

Конкретный пример написания программы на языке Otcl будет рассмотрен в Главе 3, а сейчас предлагается рассмотреть суть каждого этапа в теории.

\subsection*{Конфигурация параметров сети и узлов}

Этап конфигурирования сети необходим, чтобы определить все правила, по которым будет протекать симуляция. Этот этап включает в себя настройку следующих параметров~\cite{ns2_docs}:

\begin{enumerate}
  \item Simulation area - Область моделирования. Определяет размеры географической области, в которой будет происходить симуляция VANET. Важен для определения условий, в которых будут функционировать транспортные средства, и для реалистичности моделирования движения наземных транспортных средств. Пример возможного значения: 1000m x 1000m - означает, что область моделирования представляет собой квадрат со стороной 1000 метров.
  \item MAC Type - Тип MAC (Media Access Control - Управление доступом к среде). Описывает метод управления доступом к среде передачи данных, что критически важно для эффективной и справедливой передачи данных между узлами в сетевых условиях, характерных для VANET. Пример возможного значения: 802.11, это выбор протокола беспроводного доступа, используемого в сетях Wi-Fi, который является общепринятым для VANET.
  \item N/W Interface Type - Тип сетевого интерфейса. Определяет особенности и характеристики сетевого интерфейса, используемого узлами для обмена данными. Имеет значение для точности моделирования взаимодействия между устройствами сети. Пример возможного значения: Phy/WirelessPhy, выбор указывает на использование физического уровня беспроводной сети.
  \item Interface Queue Type - Тип очереди интерфейса. Указывает на способ управления очередями пакетов в сетевом интерфейсе, что, в свою очередь, влияет на производительность сети и качество обслуживания. Пример возможного значения: Queue/DropTail. Этот тип означает использование очереди с \textquote{отбрасыванием по принципу последний пришел - первый ушел}, которая подразумевает, что новые пакеты будут отбрасываться, если в очереди нет свободного места.
  \item Propagation model - Модель распространения. Имитирует способ распространения радиосигналов в среде, что несет критическую важность для обеспечения реалистичности моделирования коммуникационной среды в VANET. Пример возможного значения: TwoRayGround. Модель распространения \textquote{Двухлучевая наземная} учитывает как прямую линию видимости, так и отражение от земли.
  \item Transmission range - Дальность передачи. Определяет максимальное расстояние, на котором принимающий узел может успешно обнаруживать передаваемые сигналы. Этот параметр существенно влияет на структуру и свойства сети. Пример возможного значения: 250m.
  \item Antenna model - Модель антенны. Описывает характеристики антенны, используемой для передачи и приема радиосигналов, что определяет эффективность связи внутри сетевого пространства. Пример возможного значения: OmniAntenna - омнидирекциональная антенна, обеспечивающая равномерное распространение сигнала во всех направлениях.
  \item Number of nodes - Количество узлов (транспортных средств). Определяет число участвующих в сети VANET транспортных средств, что критически важно для изучения масштабируемости и производительности сети.
  \item Routing protocols - Протоколы маршрутизации. Определяют алгоритмы построения маршрутов между узлами, что влияет на эффективность и надежность коммуникации в сети. Пример возможного значения: AODV - протокол маршрутизации, рассмотренный выше в Главе 1.
  \item Transport protocols - Транспортные протоколы. Указывают на протоколы уровня транспорта, используемые для контроля передачи данных; важны для обеспечения достоверности, управления потоком и устранения ошибок в процессе передачи данных между узлами. Пример возможного значения: TCP или UDP. Выбор зависит от необходимости контроля над ошибками и потоком данных; TCP обеспечивает доставку без ошибок, в то время как UDP обеспечивает более быструю передачу без подтверждения получения.
  \item Traffic Type - Тип трафика. Классифицирует виды трафика (например, голосовой, видео, передача файлов), что позволяет более точно моделировать поведение сети под различными видами нагрузки. Пример возможного значения: FTP, означает, что будет использоваться передача файлов.
  \item Packet size - Размер пакета. Определяет размеры передаваемых данных в байтах за одну операцию отправки, что непосредственно влияет на производительность сети и эффективность использования канала передачи данных.
  \item Transmission rate - Скорость передачи. Устанавливает максимально возможную скорость передачи данных между узлами сети в битах в секунду, что является ключевым параметром для анализа пропускной способности сети.
  \item Simulation time - Время симуляции. Определяет продолжительность симуляционного эксперимента в секундах, что важно для оценки динамических характеристик и стабильности сетевых процессов на протяжении времени.
\end{enumerate}

Некоторые из этих параметров необходимо явно задавать, а какие-то из параметров подходят для симуляции в своём значении по умолчанию. Бывают параметры, которые сложно задать явно и это зачастую делается косвенно за счёт изменения логики работы элементов сети. Например - параметр дальности передачи может контролироваться мощностью передатчика узла, которую можно явно задать. Уменьшив мощность передатчиков - понизится и расстояние, на котором узлы смогут взаимодействовать друг с другом.

\subsection*{Перемещение узлов сети}

В работе рассматривается не произвольная симуляция сети в NS-2, а конкретно симуляция мобильных самоорганизующихся сетей - VANET. Соответственно, важным этапом симуляции является перемещение узлов. Как уже было описано ранее, по сути своей любое перемещение узлов - это какое-то событие в NS-2. Принцип достаточно прост - создаётся узел специальной командой, узлу задаются начальные координаты.

Для перемещения используется специальная команда setdest, которая выполняет перемещение узла в конкретную точку в конкретный момент времени.

Таким образом необходимо сформировать список команд, перемещающих узлы в определённые моменты времени в определённые точки, задавая этим скорость узлов и их мобильность. В общем и целом это делается на этапе моделирования мобильности, как уже было рассмотрено.

\subsection*{Генерация трафика в сети}

Этот этап может быть устроен по разному в зависимости от требуемых результатов. В случае рассматриваемого в работе эксперимента - предлагается генерировать трафик из одного стационарного узла в другой.

В NS-2 для этих целей используются агенты. В контексте моделирования сетей в симуляторе NS-2, агенты являются критически важными компонентами, представляющими конечные точки коммуникации или источники трафика внутри сетевой модели. Они инкапсулируют различные протоколы соединения, предоставляя механизмы для инициации, управления и завершения коммуникационных сессий. Агенты в NS-2 делятся на различные категории в зависимости от их функций и протоколов, к которым они относятся. Конкретно, Agent/TCP/Newreno и Agent/TCPSink представляют функционал для установки TCP соединения между узлами сети.

Agent/TCP/Newreno в NS-2 представляет собой реализацию протокола передачи данных TCP с использованием алгоритма управления перегрузкой сети по методу NewReno. NewReno является модификацией алгоритма Reno, предоставляя улучшенное восстановление после потери пакетов, что особенно важно в динамичных сетевых условиях, характерных для VANET. Агент выполняет задачи по надежной передаче данных, корректному управлению размером окна перегрузки и быстрому восстановлению после обнаружения потери пакетов.

Агент Agent/TCPSink, в свою очередь, предназначен для работы в паре с агентом TCP в качестве получателя данных. Он действует как конечная точка, которая принимает данные, отправляемые TCP-агентом, и обрабатывает подтверждения (ACKs) для отправленных пакетов. Это позволяет симулировать полноценный цикл TCP-соединения, включая механизмы управления перегрузкой и управления потоком. TCPSink играет важную роль в обеспечении точности моделирования поведения сетевого протокола, позволяя изучать различные аспекты работы и эффективности TCP в условиях, близких к реальности.

Чтобы установить TCP соединение между двумя узлами необходимо командой attach-agent прикрепить к двум узлам агенты Agent/TCP/Newreno и Agent/TCPSink для отправителя и получателя соответственно. Затем вызывается команда connect между двумя этими агентами.

Но генерации трафика не проиходит. Для этого необходимо поверх TCP отправлять какой-то набор данных. Для этих целей используются другого рода агенты - Application агенты.

Примеры таких агентов:

\begin{enumerate}
  \item Application/Telnet - Имитирует трафик от приложения Telnet, которое часто используется для удаленного доступа к серверам. Трафик Telnet характеризуется нерегулярными порывами данных, в зависимости от активности пользователя.
  \item Application/Ping - Эмулирует генерацию трафика эхо-запросов ICMP (Internet Control Message Protocol), как при использовании команды ping для проверки доступности хоста на сетевом соединении. Может использоваться для измерения задержек в сети и потери пакетов.
  \item Application/FTP - служит для моделирования поведения протокола передачи файлов (File Transfer Protocol - FTP) в рамках сетевого трафика. Протокол FTP используется для передачи файлов между клиентом и сервером в сети Интернет или локальной сети. В контексте симуляции сетей NS-2, агент Application/FTP позволяет исследователям точно эмулировать процесс передачи файлов и оценивать влияние различных сетевых параметров и стратегий на эффективность и надёжность этого процесса.
\end{enumerate}

Предлагается использвоать в эксперименте агент Application/FTP. Чтобы начать генерацию трафика, необходимо прикрепить агент Application/FTP к инстансу агента Agent/TCP/Newreno, который является источником данных. Чтобы начать генерацию трафика поверх TCP, необходимо использовать команду start.

\subsection*{Отслеживание событий и запись их в трейс-файлы}

Этот этап легко реализуется в NS-2 встроенными средствами. Существует команда, начинающая запись всех событий передачи данных в трейс-файлы - trace-all.
Кроме того, для создания анимации так же требуется запись в определённом формате данных о симуляции в отдельный трейс-файл. Для этих целей используются две команды: namtrace-all и namtrace-all-wireless.

Вызов этих трёх команд даст на выходе из симуляции два трейс-файла.

\subsection*{Завершение симуляции}

Этап завершения симуляции представляет собой технические действия по завершению всех процессов в симуляции и записи всех необходимых трейс-файлов. В целом теоретических знаний тут не требуется, только конкретные команды, которые будут рассмотрены в Главе 3.