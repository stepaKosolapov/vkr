\chapter{Моделирование сети VANET}

\section{Описание модели}

\subsection*{Суть эксперимента}

В данной работе будет проведён эксперимент, в рамках которого будет производится моделирование сети VANET. 

Схема эксперимента показана на рис.~\ref{fig:experiment_scheme}.

\begin{figure}[!h]
    \centering
    \includegraphics[width=1\linewidth]{"experiment.png"}
    \caption{Схема эксперимента}
    \label{fig:experiment_scheme}
\end{figure}

На какой-то местности расположены несколько стационарных узлов с беспроводным передатчиком связи (RSU). Мощность датчиков и расстояние между узлами таковы, что стационарные узлы не могут достать друг до друга и передать какие-либо данные непосредственно. 

Моделируемая местность имеет автомобильную дорогу, по которой передвигаются подвижные узлы с бортовым беспроводным передатчиком связи (OBU). Подвижные узлы передвигаются по правилам автомобильного дорожного движения на моделируемом участке дороги.

В какой-то момент симуляции, когда подвижные узлы изменят топологию сети таким образом, что между стационарными узлами образуется путь(включающий подвижные узлы), начнётся передача данных из одного стационарного узла в другой. При этом путь, по которому следуют данные, будет неизбежно меняться с течением времени симуляции, т.к. подвижные узлы почти непрерывно передвигаются и меняют топологию сети. Подвижные узлы могут так же пропадать из зоны доступности стационарных узлов.

В такой нестабильной сети, когда топология сети постоянно \textquote{рвётся}, предлагается проверить эффективность трёх протоколов маршрутизации: AODV, DSDV, DSR. Принципы моделирования сети и анализа эффективности протоколов будут рассмотрены в этой главе.

Развертывание и тестирование сетей VANET связаны с высокими затратами и требуют значительных усилий. В качестве альтернативного решения, имитационное моделирование является полезной и менее затратной альтернативой перед фактической реализацией. Для достижения хороших результатов при моделировании VANET необходимо разработать точные модели, что является непростой задачей ввиду сложностей инфраструктуры VANET (например, симуляторам необходимо моделировать как образцы мобильности, так и протоколы коммуникации)~\cite{kathiriya2013traffic, raj2014simulation, hassan2009vanet}.

Подробно процесс моделирования показан в статье ~\cite{seema2022simulation}. Автор использует набор инструментов для моделирования NS-2, SUMO и OSM и это окружение кажется достаточно удобным и быстрым в освоении и настройке.

Моделирование V2V взаимодействия между узлами сети VANET подробно проанализарован в ~\cite{mahdi2021performance}. Cравниваются три протокола маршрутизации AODV, DSR и DSDV. Так же автор объясняет разницу между этими протоколами и особенности их работы в условиях подвижных узлов сети, свойственных для VANET.

Можно сформировать общую схему имитационной модели сети VANET, которую разделим на две части: моделирование мобильности и моделирование сети.

Первая часть - моделирование мобильности, включает в себя определение количества узлов сети и описание их местоположения в каждый момент времени.
Вторая часть - моделирование сети, содержит определение конфигурации сети и описывание сетевых взаимодействий узлов в каждый момент времени симуляции.

При наложении смоделированной сети на мобильность узлов получается модель сети VANET.

Общая схема моделирования сети изображена на рис.~\ref{fig:model} и будет рассмотрена подробнее далее.

\begin{figure}[!h]
    \centering
    \includegraphics[width=1\linewidth]{"model.png"}
    \caption{Схема моделирования}
    \label{fig:model}
\end{figure}

\subsection*{Моделирование мобильности в сетях VANET}

Критически важным аспектом в исследовании симуляции сетей VANET является необходимость в модели мобильности, которая соответствовала бы реальному поведению транспортных средств в дорожном движении. Симуляторы мобильности в основном используются для генерирования закономерностей движения транспортных средств по определённому маршруту. В контексте моделирования мобильности транспортных средств различают макромобильность и микромобильность. При моделировании макромобильности симуляторам необходимо учитывать все макроскопические аспекты, влияющие на дорожное движение: топологию дорог, ограничения движения автомобилей, ограничения скорости, количество полос, правила безопасности и знаки дорожного движения, регулирующие правила проезда на перекрестках.
Микромобильность, с другой стороны, относится к индивидуальному поведению водителей при взаимодействии с другими участниками дорожного движения или с дорожной инфраструктурой: скорость движения в различных дорожных условиях, ускорение, замедление и критерии обгона, поведение на перекрёстках и при наличии дорожных знаков, общее отношение к вождению, связанное с возрастом, полом или настроением водителя. Идеальная симуляция в сетях VANET должна учитывать как описания макро-, так и микромобильности. Примеры симуляторов мобильности включают SUMO, VISSIM, SimMobility, PARAMICS и CORSIM.

В данной работе будет подробнее рассмотрен симулятор мобильности SUMO.

\subsection*{Симулятор городской мобильности SUMO}

SUMO — это пакет программ для моделирования трафика с открытым исходным кодом. SUMO используется для исследования различных аспектов движения - от выбора маршрута и алгоритмов работы светофоров до моделирования взаимодействия транспортных средств. Этот инструментарий применяется в различных проектах для моделирования автоматизированного вождения или стратегий управления трафиком, предлагая функции, такие как движение транспортных средств в непрерывном пространстве и дискретном времени, множественные типы транспортных средств, много-полосное движение с возможностью смены полосы, различные правила приоритета проезда, светофоры, удобный графический интерфейс, быстрое выполнение и взаимодействие с другими приложениями в реальном времени.

\subsection*{Генерация сценария}

Есть много способов создать файл симуляции, но один из самых удобных – это использовать OsmWebWizard. 

Схема, по которой может осуществляться моделирование мобильности в сетях изображена на рис.~\ref{fig:mobility_scheme}.

\begin{figure}
 \centering
 \includegraphics[
 height=10cm,
 keepaspectratio,
 ]{"mobility_scheme.png"}
 \caption{Схема моделирования мобильности}
 \label{fig:mobility_scheme}
\end{figure}

OsmWebWizard это программа, позволяющая создать файл симуляции SUMO через графический интерфейс. При этом программа предоставляет возможность просимулировать практически любое место на карте мира благодаря использованию OpenStreetMap в процессе генерации сценария.

Сама программа OsmWebWizard поставляется вместе с кодом SUMO и доступна после установки SUMO. Она представляет из себя скрипт на языке Python и легко запускается через кроссплатформенный интерпретатор Python.

\begin{figure}
 \centering
 \includegraphics[width=1\linewidth]{"WebWizard_interface.png"}
 \caption{Интерфейс программы OsmWebWizard}
 \label{fig:webwizard_interface}
\end{figure}

Для симуляции сети VANET зачастую не требуется моделировать никакие дороги кроме автомобильных и элементы городской инфраструктуры, поэтому их можно убрать с помощью графического интерфейса OsmWebWizard. 

Так же можно настраивать количество автомобилей, участвующих в симуляции и плотность трафика.

Нажатием на кнопку "Generate scenario" создаётся файл с настройками симуляции для SUMO в формате .sumocfg

\subsection*{Запуск симуляции в SUMO}

SUMO можно использовать как через графический интерфейс, так и через интерфейс командной строки. В зависимости от необходимого результата, нужно использовать тот или иной вариант. Если требуется визуально понаблюдать за симуляцией, то запускается графический интерфейс, а если есть необходимость только получить результаты симуляции, то удобнее использовать командный интерфейс.

Запуск симуляции может осуществляться с помощью импортирования файла .sumocfg, например, полученного из OsmWebWizard, что является удобным способом быстрого запуска симуляции мобильности.

\begin{figure}
    \centering
    \includegraphics[width=1\linewidth]{"sumo_interface.png"}
    \caption{Интерфейс программы SUMO}
    \label{fig:sumo_interface}
\end{figure}

Результатом запуска симуляции в SUMO является файл с описанием мобильности всех участвующих в симуляции узлов.

\subsection*{Преобразование файла мобильности}

Чтобы использовать результат симуляции SUMO, необходимо преобразовать его в формат, подходящий для NS-2(эта программа будет рассмотрена далее). Для этого можно использовать скрипт написанный на языке Python traceExporter.py, он поставляется так же вместе с SUMO.

Принцип работы этого скрипта прост. Он преобразует все записи о перемещении автомобилей в команды на языке NS-2 otcl, предварительно превращая автомобили в узлы сети. Таким образом, получаемый на выходе файл, по сути является набором команд на языке otcl, отвечающим за создание и перемещение конкретных узлов сети во времени.

После запуска скрипта получается файл mobility.tcl, который можно легко импортировать в NS-2. При импорте все команды, содержащиеся в файле mobility.tcl, выполнятся и события перемещения узлов, описанные в файле, будут зарегистрированы. При старте симуляции в каждый момент вермени будут выполняться все команды, зарегистрированные на этот момент времен и узлы будут перемещаться соответствующим образом, как это происходило в симуляции SUMO.


\subsection*{Моделирование сети}

Симулятор сети применяется для моделирования обмена сообщениями между соединенными узлами~\cite{aljabry2021survey}. В контексте VANET это обычно включает в себя бортовые устройства транспорта (OBU) и стационарные придорожные устройства (RSU) и в большинстве случаев связано с беспроводной коммуникацией. В идеале моделируются все компоненты коммуникационной системы (например, весь стек протоколов), и в конечном итоге симуляция также включает другие соответствующие метрики (например, отношение сигнал/шум, показатели ошибок пакетов). Модель сети описывает как компоненты сети, так и события. Узлы, маршрутизаторы, коммутаторы и связи являются примерами компонентов. События, в свою очередь, могут включать передачи данных и ошибки пакетов.

Для данного сценария симуляции выходные данные от симулятора сети обычно включают метрики, связанные с сетью, соединениями и устройствами. Также обычно доступны файлы трассировки. Такие файлы записывают каждое произошедшее событие в симуляции и могут быть обработаны для дальнейшего анализа. Большинство доступных симуляторов сетей базируются на дискретно-событийном моделировании. В этом подходе хранится список \textquote{ожидающих событий}, которые затем обрабатываются по порядку на каждом шаге симуляции. Некоторые события могут инициировать новые. Например, прибытие пакета на узел может спровоцировать отправку нового пакета. Примеры доступных симуляторов сетей (некоторые из них широко используются в сетях VANET) включают OMNeT++, OPNET, JiST/SWANS, NS-3 и NS-2.

В данной работе подробнее будет рассмотрен сетевой симулятор NS-2.

\subsection*{Сетевой симулятор NS-2}

Network Simulator 2 (NS-2) ~\cite{issariyakul2009introduction}, ~\cite{mahrenholz2004real} представляет собой программный комплекс на основе объектно-ориентированной парадигмы, разработанный для моделирования различных аспектов сетевых систем и процессов. Сердцевиной системы является ядро, написанное на языке программирования C++, а для сценариев и интерфейса пользователя применяется язык Object oriented Tool Command Language (OTcl). Наличие классовой структуры и иерархии, которая поддерживается обоими языками, позволяет удобно организовать моделирование, сохраняя при этом однозначное соответствие между элементами разработки на разных уровнях.

Платформа NS-2 обеспечивает поддержку широкого круга сетевых протоколов, что делает её весьма полезным инструментом в изучении и анализе современной сетевой инфраструктуры. Включает в себя реализации протоколов на всех уровнях сетевого обмена данных, таких как MPLS, IPv6, OSPF, и RSVP, а также несколько механизмов управления очередями (например, RED, WFQ, CBQ, SFQ), что позволяет имитировать работу реальных сетевых условий. Особое внимание уделено поддержке протоколов для беспроводных сетей, в том числе AODV, DSDV, DSR, расширяя область применения инструмента.

Среди ключевых особенностей NS-2 выделяется его гибкость в имитации различных типов трафика, включая трафик с Пуассоновским распределением и самоподобный трафик. Эта возможность дополнена способностью пользователя создавать собственные математические модели и функции, используя C++. К тому же, NS-2 позволяет моделировать ошибки на канальном уровне в процессе передачи данных, такие как искажение и потери пакетов, предлагая разнообразные методы для задания характеристик ошибок.

Для анализа и интерпретации результатов моделирования в NS-2 интегрированы инструменты визуализации, включая Network Animator (NAM) и Xgraph. NAM позволяет наглядно представить динамику сетевого взаимодействия, включая топологию, потоки данных и работы сетевого оборудования, на основе данных трассировки. Xgraph же облегчает анализ статистики сетевой активности, позволяя строить графики непосредственно изскриптов моделирования.

Программный комплекс NS-2, разрабатываемый как программное обеспечение с открытым исходным кодом, доступен для бесплатного использования, модификации и распространения. Это обеспечивает его высокую приспособляемость к специфическим исследовательским задачам и различным сетевым сценариям. Поддержка множества операционных систем, включая Linux, OS X, Solaris, FreeBSD и Windows, увеличивает удобство использования NS-2 в различных средах разработки и обеспечивает широкую доступность для пользователей, независимо от их предпочтений в операционных системах.

Важным аспектом, способствующим популярности и функциональному развитию NS-2, является активное сообщество разработчиков и пользователей. Это сообщество обеспечивает постоянное обновление и расширение возможностей программного комплекса, разработку новых модулей и протоколов, а также предоставление документации, руководств и разнообразных учебных материалов. Такой открытый подход способствует повышению качества и эффективности как самого программного обеспечения, так и проводимых с его помощью исследований.

\subsection*{Принцип моделирования сети в симуляторе NS-2}

В создании сценария симуляции в NS-2 можно выделить несколько этапов и частей.

\begin{enumerate}
  \item Конфигурация параметров сети и узлов
  \item Перемещение узлов сети
  \item Генерация трафика в сети
  \item Отслеживание событий и запись их в трейс-файлы
  \item Завершение симуляции
\end{enumerate}

Конкретный пример написания программы на языке Otcl будет рассмотрен в Главе 3, а сейчас предлагается рассмотреть суть каждого этапа в теории.

\subsection*{Конфигурация параметров сети и узлов}

Этап конфигурирования сети необходим, чтобы определить все правила, по которым будет протекать симуляция. Этот этап включает в себя настройку следующих параметров~\cite{ns2_docs}:

\begin{enumerate}
  \item Simulation area - Область моделирования. Определяет размеры географической области, в которой будет происходить симуляция VANET. Важен для определения условий, в которых будут функционировать транспортные средства, и для реалистичности моделирования движения наземных транспортных средств. Пример возможного значения: 1000m x 1000m - означает, что область моделирования представляет собой квадрат со стороной 1000 метров.
  \item MAC Type - Тип MAC (Media Access Control - Управление доступом к среде). Описывает метод управления доступом к среде передачи данных, что критически важно для эффективной и справедливой передачи данных между узлами в сетевых условиях, характерных для VANET. Пример возможного значения: 802.11, это выбор протокола беспроводного доступа, используемого в сетях Wi-Fi, который является общепринятым для VANET.
  \item N/W Interface Type - Тип сетевого интерфейса. Определяет особенности и характеристики сетевого интерфейса, используемого узлами для обмена данными. Имеет значение для точности моделирования взаимодействия между устройствами сети. Пример возможного значения: Phy/WirelessPhy, выбор указывает на использование физического уровня беспроводной сети.
  \item Interface Queue Type - Тип очереди интерфейса. Указывает на способ управления очередями пакетов в сетевом интерфейсе, что, в свою очередь, влияет на производительность сети и качество обслуживания. Пример возможного значения: Queue/DropTail. Этот тип означает использование очереди с \textquote{отбрасыванием по принципу последний пришел - первый ушел}, которая подразумевает, что новые пакеты будут отбрасываться, если в очереди нет свободного места.
  \item Propagation model - Модель распространения. Имитирует способ распространения радиосигналов в среде, что несет критическую важность для обеспечения реалистичности моделирования коммуникационной среды в VANET. Пример возможного значения: TwoRayGround. Модель распространения \textquote{Двухлучевая наземная} учитывает как прямую линию видимости, так и отражение от земли.
  \item Transmission range - Дальность передачи. Определяет максимальное расстояние, на котором принимающий узел может успешно обнаруживать передаваемые сигналы. Этот параметр существенно влияет на структуру и свойства сети. Пример возможного значения: 250m.
  \item Antenna model - Модель антенны. Описывает характеристики антенны, используемой для передачи и приема радиосигналов, что определяет эффективность связи внутри сетевого пространства. Пример возможного значения: OmniAntenna - омнидирекциональная антенна, обеспечивающая равномерное распространение сигнала во всех направлениях.
  \item Number of nodes - Количество узлов (транспортных средств). Определяет число участвующих в сети VANET транспортных средств, что критически важно для изучения масштабируемости и производительности сети.
  \item Routing protocols - Протоколы маршрутизации. Определяют алгоритмы построения маршрутов между узлами, что влияет на эффективность и надежность коммуникации в сети. Пример возможного значения: AODV - протокол маршрутизации, рассмотренный выше в Главе 1.
  \item Transport protocols - Транспортные протоколы. Указывают на протоколы уровня транспорта, используемые для контроля передачи данных; важны для обеспечения достоверности, управления потоком и устранения ошибок в процессе передачи данных между узлами. Пример возможного значения: TCP или UDP. Выбор зависит от необходимости контроля над ошибками и потоком данных; TCP обеспечивает доставку без ошибок, в то время как UDP обеспечивает более быструю передачу без подтверждения получения.
  \item Traffic Type - Тип трафика. Классифицирует виды трафика (например, голосовой, видео, передача файлов), что позволяет более точно моделировать поведение сети под различными видами нагрузки. Пример возможного значения: FTP, означает, что будет использоваться передача файлов.
  \item Packet size - Размер пакета. Определяет размеры передаваемых данных в байтах за одну операцию отправки, что непосредственно влияет на производительность сети и эффективность использования канала передачи данных.
  \item Transmission rate - Скорость передачи. Устанавливает максимально возможную скорость передачи данных между узлами сети в битах в секунду, что является ключевым параметром для анализа пропускной способности сети.
  \item Simulation time - Время симуляции. Определяет продолжительность симуляционного эксперимента в секундах, что важно для оценки динамических характеристик и стабильности сетевых процессов на протяжении времени.
\end{enumerate}

Некоторые из этих параметров необходимо явно задавать, а какие-то из параметров подходят для симуляции в своём значении по умолчанию. Бывают параметры, которые сложно задать явно и это зачастую делается косвенно за счёт изменения логики работы элементов сети. Например - параметр дальности передачи может контролироваться мощностью передатчика узла, которую можно явно задать. Уменьшив мощность передатчиков - понизится и расстояние, на котором узлы смогут взаимодействовать друг с другом.

\subsection*{Перемещение узлов сети}

В работе рассматривается не произвольная симуляция сети в NS-2, а конкретно симуляция мобильных самоорганизующихся сетей - VANET. Соответственно, важным этапом симуляции является перемещение узлов. Как уже было описано ранее, по сути своей любое перемещение узлов - это какое-то событие в NS-2. Принцип достаточно прост - создаётся узел специальной командой, узлу задаются начальные координаты.

Для перемещения используется специальная команда setdest, которая выполняет перемещение узла в конкретную точку в конкретный момент времени.

Таким образом необходимо сформировать список команд, перемещающих узлы в определённые моменты времени в определённые точки, задавая этим скорость узлов и их мобильность. В общем и целом это делается на этапе моделирования мобильности, как уже было рассмотрено.

\subsection*{Генерация трафика в сети}

Этот этап может быть устроен по разному в зависимости от требуемых результатов. В случае рассматриваемого в работе эксперимента - предлагается генерировать трафик из одного стационарного узла в другой.

В NS-2 для этих целей используются агенты. В контексте моделирования сетей в симуляторе NS-2, агенты являются критически важными компонентами, представляющими конечные точки коммуникации или источники трафика внутри сетевой модели. Они инкапсулируют различные протоколы соединения, предоставляя механизмы для инициации, управления и завершения коммуникационных сессий. Агенты в NS-2 делятся на различные категории в зависимости от их функций и протоколов, к которым они относятся. Конкретно, Agent/TCP/Newreno и Agent/TCPSink представляют функционал для установки TCP соединения между узлами сети.

Agent/TCP/Newreno в NS-2 представляет собой реализацию протокола передачи данных TCP с использованием алгоритма управления перегрузкой сети по методу NewReno. NewReno является модификацией алгоритма Reno, предоставляя улучшенное восстановление после потери пакетов, что особенно важно в динамичных сетевых условиях, характерных для VANET. Агент выполняет задачи по надежной передаче данных, корректному управлению размером окна перегрузки и быстрому восстановлению после обнаружения потери пакетов.

Агент Agent/TCPSink, в свою очередь, предназначен для работы в паре с агентом TCP в качестве получателя данных. Он действует как конечная точка, которая принимает данные, отправляемые TCP-агентом, и обрабатывает подтверждения (ACKs) для отправленных пакетов. Это позволяет симулировать полноценный цикл TCP-соединения, включая механизмы управления перегрузкой и управления потоком. TCPSink играет важную роль в обеспечении точности моделирования поведения сетевого протокола, позволяя изучать различные аспекты работы и эффективности TCP в условиях, близких к реальности.

Чтобы установить TCP соединение между двумя узлами необходимо командой attach-agent прикрепить к двум узлам агенты Agent/TCP/Newreno и Agent/TCPSink для отправителя и получателя соответственно. Затем вызывается команда connect между двумя этими агентами.

Но генерации трафика не проиходит. Для этого необходимо поверх TCP отправлять какой-то набор данных. Для этих целей используются другого рода агенты - Application агенты.

Примеры таких агентов:

\begin{enumerate}
  \item Application/Telnet - Имитирует трафик от приложения Telnet, которое часто используется для удаленного доступа к серверам. Трафик Telnet характеризуется нерегулярными порывами данных, в зависимости от активности пользователя.
  \item Application/Ping - Эмулирует генерацию трафика эхо-запросов ICMP (Internet Control Message Protocol), как при использовании команды ping для проверки доступности хоста на сетевом соединении. Может использоваться для измерения задержек в сети и потери пакетов.
  \item Application/FTP - служит для моделирования поведения протокола передачи файлов (File Transfer Protocol - FTP) в рамках сетевого трафика. Протокол FTP используется для передачи файлов между клиентом и сервером в сети Интернет или локальной сети. В контексте симуляции сетей NS-2, агент Application/FTP позволяет исследователям точно эмулировать процесс передачи файлов и оценивать влияние различных сетевых параметров и стратегий на эффективность и надёжность этого процесса.
\end{enumerate}

Предлагается использвоать в эксперименте агент Application/FTP. Чтобы начать генерацию трафика, необходимо прикрепить агент Application/FTP к инстансу агента Agent/TCP/Newreno, который является источником данных. Чтобы начать генерацию трафика поверх TCP, необходимо использовать команду start.

\subsection*{Отслеживание событий и запись их в трейс-файлы}

Этот этап легко реализуется в NS-2 встроенными средствами. Существует команда, начинающая запись всех событий передачи данных в трейс-файлы - trace-all.
Кроме того, для создания анимации так же требуется запись в определённом формате данных о симуляции в отдельный трейс-файл. Для этих целей используются две команды: namtrace-all и namtrace-all-wireless.

Вызов этих трёх команд даст на выходе из симуляции два трейс-файла.

\subsection*{Завершение симуляции}

Этап завершения симуляции представляет собой технические действия по завершению всех процессов в симуляции и записи всех необходимых трейс-файлов. В целом теоретических знаний тут не требуется, только конкретные команды, которые будут рассмотрены в Главе 3.

\section{Методология анализа результатов}

Как видно на рис.~\ref{fig:mobility_scheme}, после запуска симуляции NS-2 имеются два артефакта --- трейс-файл движения пакетов и файл анимации.

Каждый из них можно проанализировать соответствующим способом. 

\subsection{Визуальный анализ}

Первый и самый очевидный способ анализа --- визуальный. Вместе с NS-2 в архиве \textquote{all-in-one} поставляется программа NetAnim~\cite{netanim}. Это програма для визуального отображения процесса симуляции.

В скрипте NS-2, как уже было рассмотрено ранее, присутствует специальная команда namtrace-all, которая записывает все события в файл анимации. 

После запуска симуляции, с помощью NetAnim можно просмотреть анимацию из полученного файла в формате .nam.

Интерфейс NetAnim изображён на рис.~\ref{fig:netanim_interface}.

\begin{figure}[!h]
  \centering
  \includegraphics[width=1\linewidth]{"NetAnim_interface.jpeg"}
  \caption{Интерфейс NetAnim}
  \label{fig:netanim_interface}
\end{figure}

При запуске анимации --- подвижные узлы сети начинают передвигаться и посылать пакеты. Узлы в интерфейсе изображены в виде круга, содержащего внутри себя номер узла, который был задан в скрипте NS-2. 

Так как используется беспроводной канал, широковещательные сообщения изображены расплывающимися окружностями от узла. 

Направленные пакеты к определённому узлу изображены отрезком, перемещающимся в пространстве от источника к приёмнику.

С помощью визуального наблюдения можно быстро определить те проблемы, которые труднее обнаружить с помощью анализа трейс-файла. 

Например, наблюдая за анимацией, можно обнаружить, что прямой связи между стационарными узлами не существует и пакеты начинают передаваться только когда между стационарными узлами появляется подвижный узел.

Пример кадра из анимации можно увидеть на рис.~\ref{fig:netanim_frame}.

\begin{figure}[!h]
  \centering
  \includegraphics[width=1\linewidth]{"NetAnim_frame.jpeg"}
  \caption{Кадр из анимации NetAnim}
  \label{fig:netanim_frame}
\end{figure}

\subsection{Структура трейс-файла}

Чтобы проводить не визуальный анализ сети, необходимо разобраться в том, как устроен второй полученный артефакт --- трейс-файл.

Для примера будет рассмотрена 16 строка из трейс-файла, часть содержимого которого можно видеть на рис.~\ref{fig:ns2_tracefile}.

Это позволит разобраться в том, как устроен трейс-файл и придумать принцип его анализа~\cite{salleh2006trace}.

\begin{figure}[!h]
  \centering
  \includegraphics[width=1\linewidth]{"ns2_tracefile.png"}
  \caption{Содержимое трейс-файла NS-2}
  \label{fig:ns2_tracefile}
\end{figure}

\begin{verbatim}
r 10.000948296 \_38\_ RTR  --- 0 AODV 48 [0 ffffffff 0 800] ------- 
[0:255 -1:255 30 0] [0x2 1 1 [1 0] [0 4]] (REQUEST)
\end{verbatim}
Эта строка из трейс-файла NS-2 содержит информацию об одном пакете, который был отправлен в сеть. 
Расшифровка элементов строки:

\begin{itemize}
  \item \verb|r| --- тип события, в данном случае это отправка пакета;
  \item \verb|10.000948296| --- время отправки пакета в секундах;
  \item \verb|\_38\_| --- идентификатор узла, который отправил пакет;
  \item \verb|RTR| --- тип узла, который отправил пакет (в данном случае это маршрутизатор);
  \item \verb|---| --- идентификатор узла, который должен получить пакет (в данном случае это широковещательный адрес);
  \item \verb|0| --- идентификатор канала, по которому был отправлен пакет;
  \item \verb|AODV| --- протокол маршрутизации, который использовался для отправки пакета;
  \item \verb|48| --- размер пакета в байтах;
  \item \verb|[0 ffffffff 0 800]| --- информация о маршруте, который будет использоваться для доставки пакета. В данном случае это начальный пакет, поэтому маршрут неизвестен и все значения установлены в ноль;
  \item \verb|-------| --- информация о том, какими типами нод был пройден пакет. В данном случае это еще неизвестно, поэтому все значения установлены в ноль;
  \item \verb|[0:255 -1:255 30 0]| --- информация о маршруте, который будет использоваться для доставки пакета. В данном случае это начальный пакет, поэтому маршрут не известен и все значения установлены в ноль;
  \item \verb|[0x2 1 1 [1 0] [0 4]]| --- информация о пакете. \verb|[0x2]| --- тип пакета, \verb|[1]| --- идентификатор пакета, \verb|[1]| --- количество попыток отправки, \verb|[1 0]| --- номер сегмента, который был отправлен, \verb|[0 4]| --- номер сегмента, который будет отправлен;
  \item \verb|(REQUEST)| --- тип пакета, в данном случае это запрос.
\end{itemize}

Из трейс-файла можно понять состояние каждого пакета в любой момент времени симуляции: размер пакета, пункт назначения, маршрут и т.п.

Это знание позволяет провести анализ производительности сети, обработав и достав нужную информацию о пакетах из трейс-файла.

\subsection{Подсчёт параметров сети}

Чтобы посчитать параметры сети зачастую используются скрипты в формате awk. 

Принцип написания этих файлов прост, каждая строка трейс-файла разделяется на части. Как было указано выше, данные в строке трейс-файла разделены пробелами, поэтому после разделения строки пробелами получается массив данных. Каждый элемент в зависимости от типа сообщения содержит разные данные. 

Предлагается рассчитать несколько интересных для исследования параметров сети~\cite{taneja2011evaluation}:

\begin{itemize}
  \item PDR (Packet Delivery Ratio) --- процент доставки пакетов;
  \item Delay --- время, требующеся для доставки пакета внутри сети;
  \item Throughput --- пропускная способность сети;
  \item NRL (Normalized Routing Load) --- метрика, показывающая сколько пакетов из всех отправленных служит целям роутинга, а не передаче данных;
  \item Packets sent --- количество отправленных пакетов внутри сети.
\end{itemize}

\subsection{Коэффициент успешной доставки пакетов}

Коэффициент успешной доставки пакетов (PDR) определяется как отношение числа пакетов, которые были успешно доставлены к месту назначения, к числу пакетов данных, отправленных из источника. Этот показатель вычисляется путем деления количества пакетов, полученных пунктом назначения, на количество пакетов, исходящих от источника. Высокий коэффициент PDR свидетельствует о лучшей производительности протокола.

\begin{equation}
  \label{eq:pdr_calculation}
  \text{PDR} = \frac{P_r}{P_s} \cdot 100\%,
\end{equation}
где $P_r$ --- суммарное количество отправленных пакетов, а $P_s$ --- суммарное количество принятых пакетов.

\subsection{Задержка}

Среднее время задержки определяется как общее временя от отправления пакета источником до его прибытия к пункту назначения.

\begin{equation}
  \label{eq:delay_calculation}
  \text{EED} = \frac{1}{n} \sum_{i=1}^{n}(T_{r_i} - T_{s_i}) \cdot 1000,
\end{equation}
где $i$ --- идентификатор пакета, $n$ --- число успешно доставленных пакетов, $T_{r_i}$ --- время приёма пакета, а $T_{s_i}$ --- время отправки пакета. 

\subsection{Пропускная способность}

Пропускная способность определяется как число пакетов с данными, которые были успешно доставлены в сети за единицу времени. Высокая пропускная способность свидетельствует о лучшей производительности протокола.

\begin{equation}
  \label{eq:throughput_calculation}
  \text{TH} = \frac{B_r \cdot 8}{T_{\text{stop}} - T_{\text{start}} \cdot 1000},
\end{equation}
где $B_r$ --- количество полученных битов, $T_{\text{start}}$ --- время старта отправки, $T_{\text{stop}}$ --- время конца отправки.

\subsection{Метрика NRL} 

Основная цель NRL --- измерить количество управляющих данных, необходимых для доставки пакетов данных от источника к получателю, нормализуя это значение на количество успешно доставленных пакетов данных. Это позволяет провести сравнение эффективности маршрутизации между различными протоколами или конфигурациями сети относительно использования сетевых ресурсов.

\begin{equation}
  \label{eq:nrl_calculation}
  \text{NRL} = \frac{P_{s_{\text{routing}}}}{P_{r_{\text{data}}}},
\end{equation}
где $P_{s_{\text{routing}}}$ --- количество отправленных управляющих пакетов (необходимых для обеспечения маршрутизации), а $P_{r_{\text{data}}}$ --- количество успешно принятых пакетов с данными.

\subsection{Количество отправленных пакетов}

Эта метрика отвечает за количество отправленных пакетов в сети и никакой особой формулы для вычисления не имеет, достаточно подсчитывать все отправленные пакеты.

