\subsection*{Концепция VANET}

Транспортные самоорганизующиеся сети (VANET)~\cite{anwer2014survey} представляют собой подкатегорию мобильных самоорганизующихся сетей (MANET), обеспечивающих коммуникационную связь между близко расположенными транспортными средствами, а также между транспортными средствами и элементами инфраструктуры. Основой VANET являются придорожные устройства (RSU) и оборудованные транспортные средства с бортовыми устройствами (OBU). Бортовые устройства являются электронными компонентами, установленными на транспортных средствах для обеспечения взаимодействия с придорожными устройствами и другими транспортными средствами, оборудованными аналогичными устройствами. В каждом владеющем OBU транспортном средстве происходит прием сообщений от источника (другого транспортного средства или датчика), их проверка и последующая передача другим участникам дорожного движения через систему связи короткого диапазона (DSRC).

\begin{figure}[!h]
    \centering
    \includegraphics[width=1\linewidth]{"VANET_1.png"}
    \caption{Пример взаимодействия OBU и RSU}
    \label{fig:vanet}
\end{figure}

\subsection*{Основные компоненты интеллектуальных автомобилей}

В современных автомобилях интегрированы различные передовые технологии, предоставляющие услуги для мониторинга в реальном времени и обеспечения безопасности вождения. Эти устройства формируют сенсорную сеть, способную к автономной коммуникации для обмена данными.
Информация о внешней среде автомобиля собирается с помощью внутренних датчиков, которые фиксируют и сохраняют данные, такие как параметры сенсоров, навигационные данные, температуру, поведение водителя и изображения для последующего анализа. На основе этой информации автомобиль способен к автономному управлению, которое включает анализ окружающей среды, централизованное принятие решений и выполнение механических действий. Для того чтобы помочь водителям избегать дорожно-транспортных происшествий, необходимо разработать соответствующую системную архитектуру. «Умные автомобили» обычно оснащаются мультирежимными лидарами, микроволновыми радарами, камерами высокого разрешения и т.д., чтобы обеспечить точный и всесторонний сбор данных об окружающей среде. Ключевые компоненты и технологии интеллектуального автомобиля могут включать следующие элементы:

\begin{enumerate}
    \item Центральный процессор: осуществляет быструю обработку данных, выполняя арифметические, логические операции и операции ввода-вывода.
    \item Беспроводной приемопередатчик: обеспечивает передачу данных между автомобилями и между автомобилями и инфраструктурой.
    \item GPS-приемник: принимает данные от Глобальной системы позиционирования (GPS), обеспечивая навигационные услуги и точное определение местоположения транспортного средства.
    \item Датчики: размещены внутри и снаружи автомобиля для измерения скорости, расстояния до других транспортных средств и других параметров.
    \item Интерфейс ввода-вывода: обеспечивает удобную связь между человеком и автомобилем.
    \item Радар: использует радиоволны для определения расстояния и отслеживания положения близлежащих транспортных средств.
    \item ЛИДАР: датчик, использующий лазерные лучи для точного определения расстояния до объектов.
    \item OBU (On-Board Unit): внутренний блок, обеспечивающий подключение автомобиля к различным сетям и устройствам.
    \item LCS (Local Camera Sensor): датчик местного наблюдения, отслеживающий действия водителя и точно идентифицирующий объекты вокруг.
\end{enumerate}

\newpage

\subsection*{Коммуникации в VANET}

Разработка VANET становится всё более важной на фоне модернизации общественного транспорта по всему миру, как в бизнесе, так и в личных целях. В режиме реального времени передаваемая информация о состоянии дорожного покрытия или данные, касающиеся безопасности, могут транслироваться через беспроводную сеть между автомобилями или посредством других каналов, включая придорожные устройства, беспилотные летательные аппараты и т.д., способствуя предотвращению дорожных заторов и аварий. Более того, пользователи VANET могут обмениваться развлекательным контентом, вроде новостей, игр, и получать доступ в интернет, что делает длительные поездки более приятными. В результате, VANET способствует улучшению качества вождения, снижая при этом число аварий и заторов.

Для интеграции в сеть VANET транспортные средства должны быть оснащены датчиками, системами навигации, такими как GPS, мультимедийными устройствами и беспроводными модулями. Датчики и мультимедийные технологии могут применяться для обнаружения и идентификации окружающих предметов, включая другие транспортные средства, барьеры и пешеходов, что помогает избегать аварий и неожиданных сбоев. В то время как беспроводные модули расширяют возможности коммуникации, позволяя классифицировать их в зависимости от субъектов связи.

Существует несколько типов коммуникаций в VANET, включая:
\begin{enumerate}
    \item Прямую связь между транспортными средствами (V2V) без необходимости инфраструктуры.
    \item Связь между транспортными средствами и придорожной инфраструктурой (V2I/V2R), например с придорожными устройствами или базовыми станциями.
    \item Соединения между различными элементами инфраструктуры (I2I).
    \item Связь между транспортными средствами и пешеходами (V2P).
    \item Взаимодействие между транспортными средствами и дорожными барьерами (V2B).
    \item Соединения между RSU и облачными серверами (V2C).
    \item Коммуникация между транспортными средствами и беспилотными летательными аппаратами (V2U).
    \item Связь между автомобилем и датчиками (V2S) для передачи информации о дороге.
\end{enumerate}

\begin{figure}[!h]
    \centering
    \includegraphics[width=1\linewidth]{"VANET_communications.png"}
    \caption{Типы взаимодействий в VANET}
    \label{fig:vanet_communications}
\end{figure}

VANET представляет собой обширную беспроводную сеть с динамичной топологией из-за высокой мобильности транспортных средств и разнообразия маршрутов. Для оптимизации работы VANET необходимо адаптироваться к динамичной природе сетей и различным требованиям к качеству обслуживания (QoS) для предоставляемых услуг. Важно отметить, что различные приложения в рамках VANET могут иметь разнообразные требования к QoS. К примеру, приложения, не связанные непосредственно с обеспечением безопасности, могут требовать высокой пропускной способности сети, тогда как услуги, направленные на повышение безопасности, должны обеспечивать передачу данных с минимальной задержкой и высокой надёжностью.

В целом, технология VANET способствует созданию более безопасных и эффективных транспортных систем за счёт внедрения развитых средств коммуникации и обмена данными между транспортными средствами и инфраструктурой.

\newpage

\subsection*{Отличительные особенности VANET}

Сети VANET обладают уникальными характеристиками и вызовами, которые делают предоставление услуг и поддержание их надежности сложной задачей. Ниже описаны ключевые аспекты этих характеристик.

\subsection*{Динамичность мобильности}

В сетях VANET наблюдается широкий спектр мобильности участников - от стационарных элементов, таких как RSU, до медленно и быстро передвигающихся транспортных средств, что создает трудности для эффективного обмена данными внутри сети.

\subsection*{Ограничения движения}

Мобильность узлов сети VANET ограничена инфраструктурой транспортной сети, которая различается в зависимости от географического расположения и характера городской планировки, влияя тем самым на связность и производительность сети.

\subsection*{Частая фрагментация сети}

Фрагментация сети VANET обусловлена плотностью и мобильностью узлов. С уменьшением плотности сеть становится более фрагментированной, что негативно сказывается на коммуникации и доставке данных. Быстрая мобильность узлов также вносит динамику в топологию сети, разделяя её на отдельные фрагменты.

\subsection*{Неоднородность} 

Сеть VANET включает в себя разнообразные узлы с различными функциями и требованиями, варьирующиеся от стационарных элементов до мобильных транспортных средств, которые могут обмениваться как развлекательной информацией, так и критически важными данными для обеспечения безопасности движения.

\subsection*{Масштабируемость}

Сети VANET представляются способными охватывать различные географические масштабы, от малых городов до больших городских и масштабов страны, что создает проблемы в обеспечении надежности связи при масштабировании сети. Применение беспилотных летательных аппаратов для улучшения масштабируемости является одним из перспективных направлений развития.

\subsection*{Неограниченные питание и вычислительные ресурсы}

В сетях VANET отсутствуют ограничения на мощность и вычислительные возможности благодаря использованию автомобильных аккумуляторов как источника питания и интеграции OBU, что позволяет преодолевать проблемы, связанные с энергопотреблением и обработкой данных.

\subsection*{Проблема дефицита частотного спектра}

Стандарты беспроводной связи для транспортных сетей включают технологии Wireless Access in Vehicular Environments (WAVE) и Dedicated Short-Range Communications (DSRC). Однако в ходе исследований в условиях городских сетей с высокой плотностью трафика выявлены вопросы надежности и масштабируемости для сетей на базе DSRC. Основной проблемой является недостаточное количество частотных каналов для обмена данными между автомобилями, что подрывает надежность и возможность расширения сетевых систем. Транспортные средства должны конкурировать за доступ к семи доступным DSRC каналам для обмена информацией, что приводит к перегрузке сети, увеличению времени передачи данных и снижению общей пропускной способности, что отрицательно сказывается на качестве предоставляемых служб.

\subsection*{Влияние окружающей среды}

В транспортных сетях VANET коммуникация осуществляется на открытом воздухе, где воздействие окружающей среды на распространение электромагнитных сигналов может быть значительным. Различные объекты, такие как здания, автомобили и деревья, могут создавать помехи для этих сигналов, приводя к многолучевому распространению, замиранию и затуханию сигналов. Также изменение климатических условий может влиять на высокоскоростные соединения VANET, требующие передачи данных с минимальной задержкой.

\subsection*{Точность информационных данных}

Использование данных о местоположении от систем навигации, таких как GPS и GNSS, играет ключевую роль в доставке данных в среде VANET. Однако эти системы не всегда могут обеспечить высокую точность определения местоположения, особенно из-за атмосферных явлений в тропосфере и ионосфере. В городских условиях точность GPS может колебаться от 5 до 30 метров в оптимальных условиях, что недостаточно для обеспечения надежности высокоскоростных служб VANET с низкой задержкой. Кроме того, RSU собирают данные о транспортных средствах в их зоне действия, которые далее могут анализироваться облачными сервисами или системами Software-Defined Networking (SDN). Тем не менее, из-за высокой подвижности и постоянных изменений эти данные могут быть неточными, что снижает эффективность анализа и принятия решений.

\subsection*{Вопросы надежности}

Важнейшим аспектом для служб, связанных с безопасностью в сетях VANET~\cite{hahn2019security}, является обеспечение связи в режиме реального времени. Любые неточности в передаче данных могут привести к задержкам, что, в свою очередь, способно вызвать существенные проблемы на дорогах, включая серьезные заторы и аварии. Поэтому рекомендуется своевременно принять профилактические меры безопасности, чтобы предотвратить подобные инциденты, давая водителям возможность эффективно реагировать в экстренных ситуациях.

\subsection*{Обеспечение безопасности данных}

Для эффективного и надежного управления коммуникациями в сетях VANET необходимо гарантировать безопасность данных. Это включает в себя защиту данных от искажения в процессе передачи, их шифрование таким образом, чтобы предотвратить доступ третьих лиц к информации, а также подтверждение подлинности отправителя сообщений, исключая возможность подделки данных. Безопасность играет ключевую роль, поскольку при нарушении защиты системы транспортные средства могут оказаться под контролем злоумышленников, влекущих за собой дорожные пробки и аварии, потенциально угрожающие жизни. Основными трудностями при интеграции систем VANET являются обеспечение безопасности данных и подтверждение их подлинности, что усложняется из-за динамичной сетевой топологии и мобильности транспортных средств. Ключевое значение имеют методы криптографии, используемые для шифрования и дешифровки информации, требующие эффективного управления криптографическими ключами, что представляет собой сложную задачу из-за частых изменений в сетевой конфигурации VANET. Важным элементом управления ключами является их отзыв, предназначенный для нейтрализации ключей, использованных злоумышленниками, что усложняет процесс из-за роста размеров сети и увеличения количества участников.


\subsection*{Протоколы маршрутизации в сетях VANET}

Протоколы маршрутизации в сетях VANET используют технологию многоканальной беспроводной передачи информации от одного узла к другому. Данные протоколы обрабатывают информацию для маршрутизации с целью поддержания соединения, поиска оптимального пути и его сохранения в таблицах маршрутизации. Разработка протоколов маршрутизации для VANET является задачей для исследований, и на сегодняшний день исследователями и академическими работниками было предложено и проанализировано множество таких протоколов. В зависимости от техники распространения информации, метода обновления пути и пригодности для приложений, протоколы маршрутизации в VANET в основном классифицируются на основе широковещательной передачи (BBR), кластеризации (CBR), топологии (TBR), определения позиции (PBR) и географической (GBR). На рис.~\ref{fig:vanet_protocols} представлены примеры различных подходов к маршрутизации, используемых для распространения информации в VANET. Протоколы BBR применяют подходы к широковещательной рассылке для реального времени распространения информации, что полезно для приложений интеллектуальных транспортных систем, таких как сообщения о безопасности, дорожных условиях и погоде. BROADCOM, V-TRADE и DECA являются примерами протоколов BBR. Протоколы PBR предполагают, что транспортные средства имеют доступ к услугам GPS для определения своего местоположения и места назначения, используя периодическое маячение для обнаружения соседей в пределах одного прыжка и избежания столкновений.~\cite{karunakar2020analysis}

GPSR, GPCR и A-STAR являются примерами протоколов PBR. Протокол CBR разделяет сети на множество кластеров, каждый из которых состоит из узлов-участников и одного узла-главы кластера. Размер кластера зависит от техник маршрутизации, основанных на местоположении, скорости и направлении движения узла. CBDRP, CBLR и TIBCRPH являются примерами протоколов CBR. Протоколы GBR используют методы многоадресной передачи для доставки данных узлам в определенной географической области, которая называется Зоной Соответствия (ZOR). ROVER, IVG и DRG являются примерами протоколов маршрутизации GBR.

\begin{figure}[!h]
    \centering
    \includegraphics[width=1\linewidth]{"VANET_protocols.jpeg"}
    \caption{Протоколы маршрутизации VANET}
    \label{fig:vanet_protocols}
\end{figure}

\subsection*{Протоколы маршрутизации, основанные на топологии}

Протоколы маршрутизации, основанные на топологии, используют топологическую информацию, установленную в сети, для передачи данных. Эти протоколы можно разделить на три категории: прогнозирующие, реактивные и гибридные протоколы маршрутизации.

Прогнозирующие протоколы маршрутизации регулярно обновляют и поддерживают информацию о маршруте, даже когда маршрут не требуется. Они используют алгоритм кратчайшего пути для поиска маршрута. К примерам прогнозирующих протоколов маршрутизации относятся протоколы маршрутизации по векторам расстояний с последовательностью назначения (DSDV), оптимизированная маршрутизация на основе состояния каналов (OLSR) и маршрутизация с учетом "рыбьего глаза" (FSR).

Реактивные протоколы маршрутизации – это протоколы маршрутизации по требованию, которые поддерживают только недавно использованные пути маршрутизации. Эти протоколы не обновляют все маршруты постоянно. Такой подход позволяет экономить пропускную способность и снижать требования к памяти. В качестве примеров реактивных протоколов маршрутизации можно привести протокол маршрутизации по вектору расстояния "по требованию" (AODV) и динамическую маршрутизацию на основе источника (DSR).

Гибридные протоколы маршрутизации сочетают в себе свойства прогнозирующих и реактивных протоколов маршрутизации. Протокол зонной маршрутизации (ZRP) является примером гибридного протокола маршрутизации.

\subsection*{Проактивные протоколы маршрутизации}

В данном типе маршрутизации информация обо всех ассоциированных узлах сохраняется в виде таблиц. Эта таблица регулярно обновляется при каждом изменении топологии за счет обмена управляющими сообщениями. Маршруты определяются на основе этой таблицы. Проактивные протоколы не сталкиваются с начальной задержкой при поиске маршрута, но потребляют значительный объем канала связи для периодического обновления данных о топологии.

Одним из таких протоколов является протокол DSDV.

\subsection*{Протокол DSDV}

Протокол маршрутизации DSDV основан на алгоритме Беллмана-Форда и является табличным протоколом маршрутизации. Его разработали Правин Бхагват и Чарльз Э. Перкинс в 1994 году как протокол маршрутизации для мобильных ad-hoc сетей (MANET). Все узлы в сети поддерживают свою таблицу маршрутизации, которая содержит запись о назначении, необходимом количестве прыжков для достижения узла назначения и порядковом номере. Таблица маршрутизации DSDV имеет порядковый номер, который используется для предотвращения петель маршрутизации. Этот порядковый номер может быть четным или нечетным в зависимости от доступности связи. Каждый узел использует либо периодические обновления, либо методы обновления по событию для обновления таблицы маршрутизации. Обновления по событию используются каждый раз, когда узел получает пакет DSDV, вызывающий изменения в таблице маршрутизации. Поскольку в DSDV вся информация о маршруте уже доступна, нет необходимости в поиске маршрута, что приводит к меньшей задержке. Однако, поскольку в DSDV вся информация о маршруте содержится в таблице маршрутизации, если топология динамична из-за скорости узлов и увеличения размера сети, DSDV потребляет больше пропускной способности.

\subsection*{Реактивные протоколы маршрутизации}

Реактивные протоколы маршрутизации работают по принципу "по требованию", то есть маршруты между узлами сети устанавливаются только по запросу. До начала обмена пакетами между источником и пунктом назначения узел инициирует процедуру поиска маршрута, рассылая в сеть управляющий пакет для поиска кратчайшего пути к пункту назначения. Таким образом, данный тип протоколов потребляет меньше пропускной способности по сравнению с проактивными протоколами.

\subsection*{Протокол AODV}

Протокол маршрутизации AODV является реактивным и обеспечивает возможность самоорганизации и многопрыжковой связи между мобильными узлами в сетях ad-hoc. Протокол использует порядковые номера назначения для обеспечения маршрутизации без петель. Данный порядковый номер формируется узлом назначения и включается в любую отправляемую информацию о маршруте для указания узлам. Если существуют два пути к узлу назначения, выбирается путь с наибольшим порядковым номером. Благодаря реактивному подходу AODV потребляет мало пропускной способности и памяти.~\cite{pande2021performance}

Поиск маршрута в AODV начинается в порядке запрос-ответ. Исходный узел транслирует сообщение с запросом маршрута (RREQ) всем своим соседям в пределах одного прыжка для его обнаружения. Соседние узлы переотправляют это сообщение RREQ, пока оно не достигнет узла назначения. В ответ узел назначения посылает сообщение с ответом на запрос маршрута (RREP) по обратному маршруту через промежуточные узлы до исходного узла. Сообщение о ошибке маршрута (RERR) транслируется узлом, который потерял свой путь к следующему прыжку.

\subsection*{Протокол DSR}

DSR является реактивным протоколом и работает аналогично AODV. Однако отличие заключается в том, что AODV сохраняет в своей таблице маршрутизации только путь до следующего перехода, в то время как DSR сохраняет в своей таблице маршрутизации полный путь от источника к пункту назначения. DSR начинает поиск маршрута только по требованию. Исходный узел определяет полную последовательность узлов, через которые должен пройти пакет, и вставляет эту последовательность в каждый отправляемый пакет данных. Эта последовательность используется каждым узлом в маршруте для определения следующего реле до пункта назначения.