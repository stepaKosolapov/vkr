\chapter{Введение в ИТС. Сети VANET}

\section{Интеллектуальные транспортные системы}

\subsection*{Определение и классификация ИТС}

Интеллектуальные транспортные системы (ИТС)~\cite{qureshi2013survey}, будучи относительно молодой областью исследований, претерпевают трансформации с новыми идеями и инновациями с быстрыми темпами. Умные автомобили, несмотря на то что являются чрезвычайно важной составляющей ИТС, не являются единственным элементом этих систем, и существуют иные компоненты ИТС, которые получили относительно меньше внимания.

\begin{figure}[!h]
    \centering
    \includegraphics[width=1\linewidth]{"ITS_1.png"}
    \caption{Интеллектуальные транспортные системы}
    \label{fig:its}
\end{figure}

\subsection*{Умные автомобили}

Интеллектуальные транспортные системы (ИТС) часто ассоциируют с умными транспортными средствами, которые могут быть оснащены системами помощи водителю, быть полуавтономными или даже полностью автономными. Умные автомобили являются ключевым элементом ИТС из-за значительного количества личных автомобилей на дорогах. По мере развития транспортной отрасли инновации в области умных автомобилей оказывают наибольшее влияние на способы перемещения людей и базовую инфраструктуру транспортных систем. Как показано на рис.~\ref{fig:its}, умные автомобили могут формировать «колонны» с использованием технологии общения между транспортными средствами (V2V), чтобы повысить эффективность передвижения. Данная технология общения и группировки автомобилей стала возможной благодаря исследованиям в области спонтанных сетей для транспортных средств (VANET). VANET представляет собой сеть связи, организованную с использованием беспроводных устройств связи между транспортными средствами, что позволяет обмениваться данными, такими как информация об экстренных ситуациях, расстояние между автомобилями и так далее, для повышения безопасности и эффективности транспортных систем. Система VANET будет рассмотрена позже в данной работе.

Помимо взаимодействия умных автомобилей друг с другом, необходимо также обеспечить коммуникацию между компонентами внутри транспортного средства, такими как электронные управляющие блоки, для реализации различных распределенных функций управления. Это внутреннее общение обеспечивается через сети внутри автомобиля, такие как сеть управления (CAN), CAN с повышенной скоростью передачи данных (CAN FD) и FlexRay.

\subsection*{Общественный транспорт}

На сегодняшний день системы общественного транспорта во многих городах являются основным средством передвижения. Маршруты железнодорожного и автобусного транспорта функционируют практически круглосуточно, обеспечивая эффективный и экономичный доступ к перемещению для горожан. Интеллектуальные транспортные системы обладают потенциалом для повышения эффективности и пропускной способности систем общественного транспорта. Умные автобусные остановки могут предоставлять ожидающим пассажирам информацию о расписании автобусов (время прибытия и отправления) и о задержках, как это показано на рис.~\ref{fig:its}. Оптимизация маршрутов с учетом реальных дорожных условий (например, пробок, вызванных другими транспортными средствами и авариями), может повысить комфорт пассажиров и сократить время в пути. Кроме того, системы автобусов и железных дорог также могут извлекать выгоду из взаимодействия с устройствами "Интернета вещей" (ИВ) в рамках интеллектуальных транспортных систем, которые передают информацию, такую как количество пассажиров, ожидающих на остановке, или пункт назначения.

\subsection*{Интернет вещей}

Смартфоны, являясь примером устройств интернета вещей, играют важную роль в ИТС, обеспечивая не только интеграцию со смарт-автомобилями (например, в целях информационно-развлекательных систем), но и подключение к другим критически важным элементам ИТС, таким как пешеходы. Во всех крупных городах пешеходы используют дорожные пути, такие как переходы, мосты и т.д. Пешеходы могут представлять значительную опасность для транспортных средств и наоборот, поэтому алгоритмы управления светофорами и маршрутизации также должны учитывать пешеходный трафик. С помощью мобильных устройств умные контроллеры движения могут быть уведомлены о пешеходах, ожидающих перехода через улицу. Аналогичным образом, благодаря мобильным устройствам, системы общественного транспорта могут быть уведомлены о пешеходах, ожидающих посадки в метро или автобус. Пока пешеходы ожидают и пользуются системами общественного транспорта, они могут получать обновления о погоде, движении, опасностях, чрезвычайных событиях и т.д. от сети датчиков и сигналов, разбросанных по всей ИТС. Устройства сенсоров и микроконтроллеры являются примерами устройств Интернета вещей, которые все чаще используются в ИТС для различных задач сенсорики и вычислений. Например, эти устройства интернета вещей используются для сбора и обработки ценных аналитических данных для использования в алгоритмах движения и маршрутизации в умных контроллерах движения. Как показано на рис.~\ref{fig:its}, устройства ИВ все чаще используются в различных аспектах ИТС, начиная от систем общественного транспорта и заканчивая парковочными знаками, которые передают информацию о наличии мест путешественникам, ищущим место для парковки. Хотя эти устройства ИВ просты и выполняют свои непосредственные задачи очень экономично и эффективно, они испытывают трудности с реализацией базовых стандартов безопасности и конфиденциальности из-за ограничений ресурсов.

\newpage

\section{Сети VANET}

\subsection{Концепция VANET}

Транспортные самоорганизующиеся сети (VANET)~\cite{anwer2014survey} представляют собой подкатегорию мобильных самоорганизующихся сетей (MANET, Mobile Ad-hoc Network), обеспечивающих коммуникационную связь между близко расположенными транспортными средствами, а также между транспортными средствами и элементами инфраструктуры. Основой VANET являются придорожные устройства (RSU, Road State Unit) и оборудованные транспортные средства с бортовыми устройствами (OBU, On-board Unit) (см. рис.~\ref{fig:vanet}). Бортовые устройства являются электронными компонентами, установленными на транспортных средствах для обеспечения взаимодействия с придорожными устройствами и другими транспортными средствами, оборудованными аналогичными устройствами. В каждом владеющем OBU транспортном средстве происходит прием сообщений от источника (другого транспортного средства или датчика), их проверка и последующая передача другим участникам дорожного движения через систему связи короткого диапазона (DSRC, Dedicated Short-Range Communications).

\begin{figure}[!h]
    \centering
    \includegraphics[width=1\linewidth]{"VANET_1.png"}
    \caption{Пример взаимодействия OBU и RSU}
    \label{fig:vanet}
\end{figure}

\subsection{Основные компоненты интеллектуальных автомобилей}

В современных автомобилях интегрированы различные передовые технологии, предоставляющие услуги для мониторинга в реальном времени и обеспечения безопасности вождения. Эти устройства формируют сенсорную сеть, способную к автономной коммуникации для обмена данными.
Информация о внешней среде автомобиля собирается с помощью внутренних датчиков, которые фиксируют и сохраняют данные, такие как параметры сенсоров, навигационные данные, температуру, поведение водителя и изображения для последующего анализа. На основе этой информации автомобиль способен к автономному управлению, которое включает анализ окружающей среды, централизованное принятие решений и выполнение механических действий. Для того чтобы помочь водителям избегать дорожно-транспортных происшествий, необходимо разработать соответствующую системную архитектуру. «Умные автомобили» обычно оснащаются мультирежимными LiDAR (Light Detection and Ranging), микроволновыми радарами, камерами высокого разрешения и т.д., чтобы обеспечить точный и всесторонний сбор данных об окружающей среде. Ключевые компоненты и технологии интеллектуального автомобиля могут включать следующие элементы~\cite{raza2019survey}:

\begin{enumerate}
    \item Центральный процессор: осуществляет быструю обработку данных, выполняя арифметические, логические операции и операции ввода-вывода.
    \item Беспроводной приемопередатчик: обеспечивает передачу данных между автомобилями и между автомобилями и инфраструктурой.
    \item GPS-приемник: принимает данные от глобальной системы позиционирования (GPS, Global Positioning System), обеспечивая навигационные услуги и точное определение местоположения транспортного средства. Так же возможно использовать ГЛОНАСС в качестве системы навигации~\cite{savelyev2018vanet}. 
    \item Датчики: размещены внутри и снаружи автомобиля для измерения скорости, расстояния до других транспортных средств и других параметров.
    \item Интерфейс ввода-вывода: обеспечивает удобную связь между человеком и автомобилем.
    \item Радар: использует радиоволны для определения расстояния и отслеживания положения близлежащих транспортных средств.
    \item LiDAR: датчик, использующий лазерные лучи для точного определения расстояния до объектов.
    \item OBU: внутренний блок, обеспечивающий подключение автомобиля к различным сетям и устройствам~\cite{truong2015software}.
    \item LCS (Local Camera Sensor): датчик местного наблюдения, отслеживающий действия водителя и точно идентифицирующий объекты вокруг.
\end{enumerate}

\subsection{Коммуникации в VANET}

Разработка VANET становится всё более важной на фоне модернизации общественного транспорта по всему миру, как в бизнесе, так и в личных целях. В режиме реального времени передаваемая информация о состоянии дорожного покрытия или данные, касающиеся безопасности, могут транслироваться через беспроводную сеть между автомобилями или посредством других каналов, включая придорожные устройства, беспилотные летательные аппараты и т.д., способствуя предотвращению дорожных заторов и аварий. Более того, пользователи VANET могут обмениваться развлекательным контентом, вроде новостей, игр, и получать доступ в интернет, что делает длительные поездки более приятными. В результате, VANET способствует улучшению качества вождения, снижая при этом число аварий и заторов.

Для интеграции в сеть VANET транспортные средства должны быть оснащены датчиками, системами навигации, такими как GPS, мультимедийными устройствами и беспроводными модулями. Датчики и мультимедийные технологии могут применяться для обнаружения и идентификации окружающих предметов, включая другие транспортные средства, барьеры и пешеходов, что помогает избегать аварий и неожиданных сбоев. В то время как беспроводные модули расширяют возможности коммуникации, позволяя классифицировать их в зависимости от субъектов связи.

Существует несколько типов коммуникаций в VANET~\cite{daniel2016cooperative} (рис.~\ref{fig:vanet_communications}):
\begin{enumerate}
    \item Прямая связь между транспортными средствами (V2V)~\cite{bintoro2021study} без необходимости инфраструктуры.
    \item Связь между транспортными средствами и придорожной инфраструктурой (V2I/V2R, Vehicles To Infrastructure/Vehicle To Roadside), например с придорожными устройствами или базовыми станциями.
    \item Соединения между различными элементами инфраструктуры (I2I).
    \item Связь между транспортными средствами и пешеходами (V2P, Vehicle To Passenger).
    \item Взаимодействие между транспортными средствами и дорожными барьерами (V2B, Vehicle To Barrier).
    \item Соединения между RSU и облачными серверами (V2C, Vehicle To Cloud).
    \item Коммуникация между транспортными средствами и беспилотными летательными аппаратами (V2U, Vehicle To UAV).
    \item Связь между автомобилем и датчиками (V2S, Vehicle To Sensor) для передачи информации о дороге.
\end{enumerate}

\begin{figure}[!h]
    \centering
    \includegraphics[width=1\linewidth]{"VANET_communications.png"}
    \caption{Типы взаимодействий в VANET}
    \label{fig:vanet_communications}
\end{figure}

VANET представляет собой обширную беспроводную сеть с динамичной топологией из-за высокой мобильности транспортных средств и разнообразия маршрутов. Для оптимизации работы VANET необходимо адаптироваться к динамичной природе сетей и различным требованиям к качеству обслуживания (QoS, Quality of Service) для предоставляемых услуг. Важно отметить, что различные приложения в рамках VANET могут иметь разнообразные требования к QoS. Например, приложения, не связанные непосредственно с обеспечением безопасности, могут требовать высокой пропускной способности сети, тогда как услуги, направленные на повышение безопасности, должны обеспечивать передачу данных с минимальной задержкой и высокой надёжностью.

В целом, технология VANET способствует созданию более безопасных и эффективных транспортных систем за счёт внедрения развитых средств коммуникации и обмена данными между транспортными средствами и инфраструктурой.

\subsection{Отличительные особенности VANET}

Сети VANET обладают уникальными характеристиками и вызовами, которые делают предоставление услуг и поддержание их надежности сложной задачей. Ниже описаны ключевые аспекты этих характеристик~\cite{hamdi2020review,karagiannis2011vehicular,raw2013security}.

\paragraph{Динамичность мобильности.}

В сетях VANET наблюдается широкий спектр мобильности участников --- от стационарных элементов, таких как RSU, до медленно и быстро передвигающихся транспортных средств, что создает трудности для эффективного обмена данными внутри сети.

\paragraph{Ограничения движения.}

Мобильность узлов сети VANET ограничена инфраструктурой транспортной сети, которая различается в зависимости от географического расположения и характера городской планировки, влияя тем самым на связность и производительность сети.

\paragraph{Частая фрагментация сети.}

Фрагментация сети VANET обусловлена плотностью и мобильностью узлов. С уменьшением плотности сеть становится более фрагментированной, что негативно сказывается на коммуникации и доставке данных. Быстрая мобильность узлов также вносит динамику в топологию сети, разделяя её на отдельные фрагменты.

\paragraph{Неоднородность.} 

Сеть VANET включает в себя разнообразные узлы с различными функциями и требованиями, варьирующиеся от стационарных элементов до мобильных транспортных средств, которые могут обмениваться как развлекательной информацией, так и критически важными данными для обеспечения безопасности движения.

\paragraph{Масштабируемость.}

Сети VANET представляются способными охватывать различные географические масштабы, от малых городов до больших городских и масштабов страны, что создает проблемы в обеспечении надежности связи при масштабировании сети. Применение беспилотных летательных аппаратов для улучшения масштабируемости является одним из перспективных направлений развития.

\paragraph{Неограниченные питание и вычислительные ресурсы.}

В сетях VANET отсутствуют ограничения на мощность и вычислительные возможности благодаря использованию автомобильных аккумуляторов как источника питания и интеграции OBU, что позволяет преодолевать проблемы, связанные с энергопотреблением и обработкой данных.

\paragraph{Проблема дефицита частотного спектра.}

Стандарты беспроводной связи для транспортных сетей включают технологии WAVE (Wireless Access in Vehicular Environments ) и DSRC. Однако в ходе исследований в условиях городских сетей с высокой плотностью трафика выявлены вопросы надежности и масштабируемости для сетей на базе DSRC. Основной проблемой является недостаточное количество частотных каналов для обмена данными между автомобилями, что подрывает надежность и возможность расширения сетевых систем. Транспортные средства должны конкурировать за доступ к семи доступным DSRC каналам для обмена информацией, что приводит к перегрузке сети, увеличению времени передачи данных и снижению общей пропускной способности, что отрицательно сказывается на качестве предоставляемых служб.

\paragraph{Влияние окружающей среды.}

В транспортных сетях VANET коммуникация осуществляется на открытом воздухе, где воздействие окружающей среды на распространение электромагнитных сигналов может быть значительным. Различные объекты, такие как здания, автомобили и деревья, могут создавать помехи для этих сигналов, приводя к многолучевому распространению, замиранию и затуханию сигналов. Также изменение климатических условий может влиять на высокоскоростные соединения VANET, требующие передачи данных с минимальной задержкой.

\paragraph{Точность информационных данных.}

Использование данных о местоположении от систем навигации, таких как GPS и GNSS (Global Navigation Satellite System), играет ключевую роль в доставке данных в среде VANET. Однако эти системы не всегда могут обеспечить высокую точность определения местоположения, особенно из-за атмосферных явлений в тропосфере и ионосфере. В городских условиях точность GPS может колебаться от 5 до 30 метров в оптимальных условиях, что недостаточно для обеспечения надежности высокоскоростных служб VANET с низкой задержкой. Кроме того, RSU собирают данные о транспортных средствах в их зоне действия, которые далее могут анализироваться облачными сервисами или системами  SDN (Software-Defined Networking). Тем не менее, из-за высокой подвижности и постоянных изменений эти данные могут быть неточными, что снижает эффективность анализа и принятия решений.

\paragraph{Вопросы надежности.}

Важнейшим аспектом для служб, связанных с безопасностью в сетях VANET~\cite{hahn2019security}, является обеспечение связи в режиме реального времени. Любые неточности в передаче данных могут привести к задержкам, что, в свою очередь, способно вызвать существенные проблемы на дорогах, включая серьезные заторы и аварии. Поэтому рекомендуется своевременно принять профилактические меры безопасности, чтобы предотвратить подобные инциденты, давая водителям возможность эффективно реагировать в экстренных ситуациях.

\paragraph{Обеспечение безопасности данных.}

Для эффективного и надежного управления коммуникациями в сетях VANET необходимо гарантировать безопасность данных. Это включает в себя защиту данных от искажения в процессе передачи, их шифрование таким образом, чтобы предотвратить доступ третьих лиц к информации, а также подтверждение подлинности отправителя сообщений, исключая возможность подделки данных. Безопасность играет ключевую роль, поскольку при нарушении защиты системы транспортные средства могут оказаться под контролем злоумышленников, влекущих за собой дорожные пробки и аварии, потенциально угрожающие жизни. Основными трудностями при интеграции систем VANET являются обеспечение безопасности данных и подтверждение их подлинности, что усложняется из-за динамичной сетевой топологии и мобильности транспортных средств. Ключевое значение имеют методы криптографии, используемые для шифрования и дешифровки информации, требующие эффективного управления криптографическими ключами, что представляет собой сложную задачу из-за частых изменений в сетевой конфигурации VANET. Важным элементом управления ключами является их отзыв, предназначенный для нейтрализации ключей, использованных злоумышленниками, что усложняет процесс из-за роста размеров сети и увеличения количества участников.


\subsection{Протоколы маршрутизации в сетях VANET}

Протоколы маршрутизации в сетях VANET используют технологию многоканальной беспроводной передачи информации от одного узла к другому. Данные протоколы обрабатывают информацию для маршрутизации с целью поддержания соединения, поиска оптимального пути и его сохранения в таблицах маршрутизации. Разработка протоколов маршрутизации для VANET является задачей для исследований, и на сегодняшний день исследователями и академическими работниками было предложено и проанализировано множество таких протоколов. В зависимости от техники распространения информации, метода обновления пути и пригодности для приложений, протоколы маршрутизации в VANET в основном классифицируются на основе широковещательной передачи (BBR, Broadcast Based Routing), кластеризации (CBR, Cluster Based Routing), топологии (TBR, Topology Based Routing), определения позиции (PBR, Position Based Routing) и географической (GBR, Geocast Based Routing). На рис.~\ref{fig:vanet_protocols} представлены примеры различных подходов к маршрутизации, используемых для распространения информации в VANET.

Протоколы BBR применяют подходы к широковещательной рассылке для реального времени распространения информации, что полезно для приложений интеллектуальных транспортных систем, таких как сообщения о безопасности, дорожных условиях и погоде. BROADCOM, V-TRADE и DECA являются примерами протоколов BBR. Протоколы PBR предполагают, что транспортные средства имеют доступ к услугам GPS для определения своего местоположения и места назначения, используя периодическое маячение для обнаружения соседей в пределах одного прыжка и избежания столкновений~\cite{karunakar2020analysis}.

GPSR, GPCR и A-STAR являются примерами протоколов PBR. Протокол CBR разделяет сети на множество кластеров, каждый из которых состоит из узлов-участников и одного узла-главы кластера. Размер кластера зависит от техник маршрутизации, основанных на местоположении, скорости и направлении движения узла. CBDRP, CBLR и TIBCRPH являются примерами протоколов CBR. Протоколы GBR используют методы многоадресной передачи для доставки данных узлам в определенной географической области, которая называется Зоной Соответствия (ZOR). ROVER, IVG и DRG являются примерами протоколов маршрутизации GBR.

\begin{figure}[!h]
    \centering
    \includegraphics[width=1\linewidth]{"VANET_protocols.jpeg"}
    \caption{Протоколы маршрутизации VANET}
    \label{fig:vanet_protocols}
\end{figure}

\subsection{Протоколы маршрутизации, основанные на топологии}

Протоколы маршрутизации, основанные на топологии, используют топологическую информацию, установленную в сети, для передачи данных. Эти протоколы можно разделить на три категории: прогнозирующие, реактивные и гибридные протоколы маршрутизации~\cite{kaur2013comparative, garg2012review}.

Прогнозирующие протоколы маршрутизации регулярно обновляют и поддерживают информацию о маршруте, даже когда маршрут не требуется. Они используют алгоритм кратчайшего пути для поиска маршрута. К примерам прогнозирующих протоколов маршрутизации относятся протоколы маршрутизации по векторам расстояний с последовательностью назначения (DSDV, Destination-Sequenced Distance Vector), оптимизированная маршрутизация на основе состояния каналов (OLSR, Optimized Link State Routing) и маршрутизация с учетом \textquote{рыбьего глаза} (FSR, Fisheye State Routing).

Реактивные протоколы маршрутизации --- это протоколы маршрутизации по требованию, которые поддерживают только недавно использованные пути маршрутизации. Эти протоколы не обновляют все маршруты постоянно. Такой подход позволяет экономить пропускную способность и снижать требования к памяти. В качестве примеров реактивных протоколов маршрутизации можно привести протокол маршрутизации по вектору расстояния \textquote{по требованию} (AODV, Ad-hoc On-Demand Distance Vector) и динамическую маршрутизацию на основе источника (DSR).

Гибридные протоколы маршрутизации сочетают в себе свойства прогнозирующих и реактивных протоколов маршрутизации. Протокол зонной маршрутизации (ZRP, Zone Routing Protocol) является примером гибридного протокола маршрутизации.

\subsection{Проактивные протоколы маршрутизации}

В данном типе маршрутизации информация обо всех ассоциированных узлах сохраняется в виде таблиц. Эта таблица регулярно обновляется при каждом изменении топологии за счет обмена управляющими сообщениями. Маршруты определяются на основе этой таблицы. Проактивные протоколы не сталкиваются с начальной задержкой при поиске маршрута, но потребляют значительный объем канала связи для периодического обновления данных о топологии.

Одним из таких протоколов является протокол DSDV.

\subsection{Протокол DSDV}

Протокол маршрутизации DSDV основан на алгоритме Беллмана--Форда и является табличным протоколом маршрутизации~\cite{perkins1994highly}. Его разработали Правин Бхагват и Чарльз Э. Перкинс в 1994 году как протокол маршрутизации для мобильных ad-hoc сетей (MANET). Все узлы в сети поддерживают свою таблицу маршрутизации, которая содержит запись о назначении, необходимом количестве прыжков для достижения узла назначения и порядковом номере. Таблица маршрутизации DSDV имеет порядковый номер, который используется для предотвращения петель маршрутизации. Этот порядковый номер может быть четным или нечетным в зависимости от доступности связи. Каждый узел использует либо периодические обновления, либо методы обновления по событию для обновления таблицы маршрутизации. Обновления по событию используются каждый раз, когда узел получает пакет DSDV, вызывающий изменения в таблице маршрутизации. Поскольку в DSDV вся информация о маршруте уже доступна, нет необходимости в поиске маршрута, что приводит к меньшей задержке. Однако, поскольку в DSDV вся информация о маршруте содержится в таблице маршрутизации, если топология динамична из-за скорости узлов и увеличения размера сети, DSDV потребляет больше пропускной способности.

\subsection{Реактивные протоколы маршрутизации}

Реактивные протоколы маршрутизации работают по принципу \textquote{по требованию}, то есть маршруты между узлами сети устанавливаются только по запросу. До начала обмена пакетами между источником и пунктом назначения узел инициирует процедуру поиска маршрута, рассылая в сеть управляющий пакет для поиска кратчайшего пути к пункту назначения. Таким образом, данный тип протоколов потребляет меньше пропускной способности по сравнению с проактивными протоколами.

\subsection{Протокол AODV}

Протокол маршрутизации AODV является реактивным и обеспечивает возможность самоорганизации и многопрыжковой связи между мобильными узлами в сетях ad-hoc. Протокол использует порядковые номера назначения для обеспечения маршрутизации без петель. Данный порядковый номер формируется узлом назначения и включается в любую отправляемую информацию о маршруте для указания узлам. Если существуют два пути к узлу назначения, выбирается путь с наибольшим порядковым номером. Благодаря реактивному подходу AODV потребляет мало пропускной способности и памяти~\cite{pande2021performance}.

Поиск маршрута в AODV начинается в порядке запрос-ответ. Исходный узел транслирует сообщение с запросом маршрута (RREQ) всем своим соседям в пределах одного прыжка для его обнаружения. Соседние узлы переотправляют это сообщение RREQ, пока оно не достигнет узла назначения. В ответ узел назначения посылает сообщение с ответом на запрос маршрута (RREP) по обратному маршруту через промежуточные узлы до исходного узла. Сообщение о ошибке маршрута (RERR) транслируется узлом, который потерял свой путь к следующему прыжку.

\subsection{Протокол DSR}

DSR (Dynamic Source Routing)~\cite{johnson2001dsr} является реактивным протоколом и работает аналогично AODV. Однако отличие заключается в том, что AODV сохраняет в своей таблице маршрутизации только путь до следующего перехода, в то время как DSR сохраняет в своей таблице маршрутизации полный путь от источника к пункту назначения~\cite{bai2006doa}. DSR начинает поиск маршрута только по требованию. Исходный узел определяет полную последовательность узлов, через которые должен пройти пакет, и вставляет эту последовательность в каждый отправляемый пакет данных. Эта последовательность используется каждым узлом в маршруте для определения следующего реле до пункта назначения.

\subsection*{Выводы}

В данной главе дипломной работы был проведен обзор литературы по теме ИТС и VANET, содержащий анализ и классификацию основных элементов, концепций и технологий, используемых в интеллектуальных транспортных системах, с особым вниманием к развитию умных автомобилей, общественного транспорта, систем коммуникаций и протоколов маршрутизации в VANET. Были описаны ключевые аспекты умных автомобилей, включая их взаимодействие через VANET и внутрисистемные сети, а также значимость интеграции интернета вещей в разработку ИТС. Далее, были рассмотрены принципы работы и основные области применения VANET, подчеркнута ее роль в улучшении безопасности и эффективности на дорогах. Протоколы маршрутизации, как жизненно важная часть сетевых коммуникаций в VANET, были описаны с акцентом на их классификации: основанные на топологии, географическом положении и различных подходах к маршрутизации. В частности, были описаны такие протоколы, как DSDV, AODV, DSR, с описанием их особенностей, преимуществ и недостатков. Помимо этого, были освещены уникальные аспекты работы сетей VANET, включая динамичность, масштабируемость и влияние окружающей среды на коммуникацию.