\subsection{Определение и классификация ИТС}

Интеллектуальные транспортные системы (ИТС)~\cite{qureshi2013survey}, будучи относительно молодой областью исследований, претерпевают трансформации с новыми идеями и инновациями с быстрыми темпами. Умные автомобили, несмотря на то что являются чрезвычайно важной составляющей ИТС, не являются единственным элементом этих систем, и существуют иные компоненты ИТС, которые получили относительно меньше внимания.

\subsection{Умные автомобили}

Интеллектуальные транспортные системы часто ассоциируют с умными транспортными средствами~\cite{i2018vehicular}, которые могут быть оснащены системами помощи водителю, быть полуавтономными или даже полностью автономными. Умные автомобили являются ключевым элементом ИТС из-за значительного количества личных автомобилей на дорогах. По мере развития транспортной отрасли инновации в области умных автомобилей оказывают наибольшее влияние на способы перемещения людей и базовую инфраструктуру транспортных систем. Как показано на рис.~\ref{fig:its}, умные автомобили могут формировать \textquote{колонны} с использованием технологии общения между транспортными средствами (V2V, Vehicles To Vehicles), чтобы повысить эффективность передвижения. Данная технология общения и группировки автомобилей стала возможной благодаря исследованиям в области спонтанных сетей для транспортных средств (VANET, Vehicular Ad-hoc Networks). 

VANET представляет собой сеть связи с использованием беспроводных устройств связи между транспортными средствами, что позволяет обмениваться данными, такими как информация об экстренных ситуациях, расстояние между автомобилями и так далее, для повышения безопасности и эффективности транспортных систем.

\begin{figure}[!h]
    \centering
    \includegraphics[width=1\linewidth]{"ITS_1.png"}
    \caption{Интеллектуальные транспортные системы}
    \label{fig:its}
\end{figure}

Помимо взаимодействия умных автомобилей друг с другом, необходимо также обеспечить коммуникацию между компонентами внутри транспортного средства, такими как электронные управляющие блоки, для реализации различных распределенных функций управления. Это внутреннее общение обеспечивается через сети внутри автомобиля, такие как сеть управления (CAN, Control Area Network), CAN с повышенной скоростью передачи данных (CAN FD, Controller Area Network Flexible Data-Rate) и FlexRay.

\subsection{Общественный транспорт}

На сегодняшний день системы общественного транспорта во многих городах являются основным средством передвижения. Маршруты железнодорожного и автобусного транспорта функционируют практически круглосуточно, обеспечивая эффективный и экономичный доступ к перемещению для горожан. Интеллектуальные транспортные системы обладают потенциалом для повышения эффективности и пропускной способности систем общественного транспорта. Умные автобусные остановки могут предоставлять ожидающим пассажирам информацию о расписании автобусов (время прибытия и отправления) и о задержках, как это показано на рис.~\ref{fig:its}. Оптимизация маршрутов с учетом реальных дорожных условий (например, пробок, вызванных другими транспортными средствами и авариями), может повысить комфорт пассажиров и сократить время в пути. Кроме того, системы автобусов и железных дорог также могут извлекать выгоду из взаимодействия с устройствами \textquote{интернета вещей} (ИВ) в рамках интеллектуальных транспортных систем, которые передают информацию, такую как количество пассажиров, ожидающих на остановке, или пункт назначения.

\subsection{Интернет вещей}

Смартфоны, являясь примером устройств интернета вещей~\cite{hatim2018vanets}, играют важную роль в ИТС, обеспечивая не только интеграцию со смарт-автомобилями (например, в целях информационно-развлекательных систем), но и подключение к другим критически важным элементам ИТС, таким как пешеходы. Во всех крупных городах пешеходы используют дорожные пути, такие как переходы, мосты и т.д. Пешеходы могут представлять значительную опасность для транспортных средств и наоборот, поэтому алгоритмы управления светофорами и маршрутизации также должны учитывать пешеходный трафик. С помощью мобильных устройств умные контроллеры движения могут быть уведомлены о пешеходах, ожидающих перехода через улицу. Аналогичным образом, благодаря мобильным устройствам, системы общественного транспорта могут быть уведомлены о пешеходах, ожидающих посадки в метро или автобус. Пока пешеходы ожидают и пользуются системами общественного транспорта, они могут получать обновления о погоде, движении, опасностях, чрезвычайных событиях и т.д. от сети датчиков и сигналов, разбросанных по всей ИТС. Устройства сенсоров и микроконтроллеры являются примерами устройств Интернета вещей, которые все чаще используются в ИТС для различных задач сенсорики и вычислений. Например, эти устройства интернета вещей используются для сбора и обработки ценных аналитических данных для использования в алгоритмах движения и маршрутизации в умных контроллерах движения. Как показано на рис.~\ref{fig:its}, устройства ИВ все чаще используются в различных аспектах ИТС, начиная от систем общественного транспорта и заканчивая парковочными знаками, которые передают информацию о наличии мест путешественникам, ищущим место для парковки. Хотя эти устройства ИВ просты и выполняют свои непосредственные задачи очень экономично и эффективно, они испытывают трудности с реализацией базовых стандартов безопасности и конфиденциальности из-за ограничений ресурсов.

\newpage